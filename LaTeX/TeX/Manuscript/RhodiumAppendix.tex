
\newpage
\section{Appendix}
\label{appendix}

\subsection{Baseline Effect of Civil Conflict on Growth}

In Table \ref{tab:baseline} below, we show the fixed effects regression results for the model in which we utilize a full country year panel in order to define a baseline effect of civil war on economic growth. 

% Table created by stargazer v.5.1 by Marek Hlavac, Harvard University. E-mail: hlavac at fas.harvard.edu
% Date and time: Tue, Aug 25, 2015 - 15:53:07
\begin{table}[!htbp] \centering 
  \caption{This table shows the results of a country fixed effects regression in which we are utilizing a full country year dataset. } 
  \label{tab:baseline} 
\footnotesize{
\begin{tabular}{@{\extracolsep{5pt}}lc} 
\\[-1.8ex]\hline 
\hline \\[-1.8ex] 
 & \multicolumn{1}{c}{\textit{Dependent variable:}} \\ 
\cline{2-2} 
\\[-1.8ex] & $\% \Delta GDP_{t}$ \\ 
\hline \\[-1.8ex] 
 Civil War$_{t-1}$ & $-$2.568$^{***}$ \\ 
  & (0.502) \\ 
  & \\ 
 Ln(Inflation)$_{t-1}$ & $-$3.040$^{***}$ \\ 
  & (0.228) \\ 
  & \\ 
 Democracy$_{t-1}$ & 0.043 \\ 
  & (0.045) \\ 
  & \\ 
 Resource Rents/GDP$_{t-1}$ & 0.115$^{***}$ \\ 
  & (0.018) \\ 
  & \\ 
 World GDP Growth$_{t}$ & 0.673$^{***}$ \\ 
  & (0.081) \\ 
  & \\ 
\hline \\[-1.8ex] 
Countries & 160 \\
Observations & 3,002 \\ 
\hline 
\hline \\[-1.8ex] 
\textit{Note:}  & \multicolumn{1}{l}{$^{*}$p$<$0.1; $^{**}$p$<$0.05; $^{***}$p$<$0.01} \\ 
\end{tabular} 
}
\end{table} 
\FloatBarrier
\newpage

\subsection{Full Panel Fixed Effects Analysis}

Here we broaden the random effect conflict-year analysis presented in the paper to a country-year analysis in which we employ a fixed effects framework. Specifically, we estimate the following equation: $\% \Delta GDP_{i,t} = \beta_{1}Conflict_{i,t-1} + \beta_{2}Distance_{i,t-1}+\ldots+\zeta_{i,t-1}$, where $Distance_{i,t-1}=0$ when $Conflict_{i,t-1}=0$ and $Distance_{i,t-1}>0$ when $Conflict_{i,t-1}=1$. The $\zeta_{i,t-1}$ term represents our bundle of control variables. This specification provides one way to consider the full sample, control for the presence of armed conflict, and test our key hypothesis regarding the role of conflict proximity to major urban centers. In this specification, we expect the estimate of $\beta_{2}$ to be positive.

The results are presented in Tables~\ref{tab:cityFullPiecewiseFE} and \ref{tab:capFullPiecewiseFE}. In the first column of these tables, we run a pooled country-year analysis in which we include our key variable of interest, distance to conflict, and controls. In the second column, we incorporate country fixed effects and in the last we utilize both country and year fixed effects. When adding year fixed effects we remove our global average GDP growth measure. Since these analyses are performed on the full panel of data we have to exclude the conflict-specific covariates from the PRIO dataset as they are undefined for countries not experiencing civil war. 

When utilizing this alternative approach we still find strong support for the hypotheses that countries experiencing conflict more proximate to capital or major cities are more likely to see significant reductions in GDP growth than conflicts that are farther away from these economically vital centers.

% Piecewise mindist fixed effects model
% Table created by stargazer v.5.2 by Marek Hlavac, Harvard University. E-mail: hlavac at fas.harvard.edu
% Date and time: Thu, Nov 05, 2015 - 23:34:34
\begin{table}[!htbp] \centering 
  \caption{This table shows the results of a series of regressions estimating the effect of a conflict distance to major cities on GDP growth in which we utilize a full country year panel. The first column shows the results of our base model with controls estimated with no fixed effects, next we add fixed effects for countries, and last we incorporate both country and year fixed effects.} 
  \label{tab:cityFullPiecewiseFE} 
\begin{tabular}{@{\extracolsep{5pt}}lccc} 
\\[-1.8ex]\hline 
\hline \\[-1.8ex] 
 & \multicolumn{3}{c}{\textit{Dependent variable:}} \\ 
\cline{2-4} 
\\[-1.8ex] & \multicolumn{3}{c}{$\% \Delta GDP_{t}$} \\ 
\\[-1.8ex] & \textit{Pooled} 
 & \textit{Country FE} & \textit{Country + Year FE} \\ 
\\[-1.8ex] & (1) & (2) & (3)\\ 
\hline \\[-1.8ex] 
 Civil War$_{t-1}$ & $-$6.899$^{***}$ & $-$8.979$^{***}$ & $-$9.145$^{***}$ \\ 
  & (1.369) & (1.690) & (1.697) \\ 
  & & & \\ 
 Ln(Min. City Dist.)$_{t-1}$ & 1.267$^{***}$ & 1.281$^{***}$ & 1.312$^{***}$ \\ 
  & (0.260) & (0.322) & (0.324) \\ 
  & & & \\ 
 Ln(Inflation)$_{t-1}$ & $-$2.908$^{***}$ & $-$2.963$^{***}$ & $-$3.156$^{***}$ \\ 
  & (0.200) & (0.228) & (0.237) \\ 
  & & & \\ 
 Democracy$_{t-1}$ & $-$0.065$^{***}$ & 0.047 & 0.081$^{*}$ \\ 
  & (0.020) & (0.045) & (0.048) \\ 
  & & & \\ 
 Resource Rents/GDP$_{t-1}$ & 0.054$^{***}$ & 0.116$^{***}$ & 0.120$^{***}$ \\ 
  & (0.010) & (0.018) & (0.019) \\ 
  & & & \\ 
 World GDP Growth$_{t}$ & 0.756$^{***}$ & 0.663$^{***}$ &  \\ 
  & (0.083) & (0.081) &  \\ 
  & & & \\ 
 % Intercept & 4.017$^{***}$ & 12.582$^{***}$ &  &  \\ 
 %  & (0.145) & (0.959) &  &  \\ 
 %  & & & & \\ 
\hline \\[-1.8ex] 
Countries & 160 & 160 & 160 \\
Observations & 3,002 & 3,002 & 3,002 \\ 
% R$^{2}$ & 0.013 & 0.140 & 0.151 & 0.095 \\ 
% Adjusted R$^{2}$ & 0.013 & 0.138 & 0.143 & 0.089 \\ 
% Residual Std. Error & 7.238 (df = 2999) & 6.763 (df = 2995) &  &  \\ 
% F Statistic & 20.180$^{***}$ (df = 2; 2999) & 81.155$^{***}$ (df = 6; 2995) & 84.088$^{***}$ (df = 6; 2836) & 58.895$^{***}$ (df = 5; 2818) \\ 
\hline 
\hline \\[-1.8ex] 
\textit{Note:}  & \multicolumn{3}{r}{$^{*}$p$<$0.1; $^{**}$p$<$0.05; $^{***}$p$<$0.01} \\ 
\end{tabular} 
\end{table} 
\FloatBarrier

% Piecewise capdist fixed effects model
% Table created by stargazer v.5.2 by Marek Hlavac, Harvard University. E-mail: hlavac at fas.harvard.edu
% Date and time: Thu, Nov 05, 2015 - 23:34:55
\begin{table}[!htbp] \centering 
  \caption{This table shows the results of a series of regressions estimating the effect of a conflict distance to capital cities on GDP growth in which we utilize a full country year panel. The first column shows the results of our base model with controls estimated with no fixed effects, next we add fixed effects for countries, and last we incorporate both country and year fixed effects.} 
  \label{tab:capFullPiecewiseFE} 
\begin{tabular}{@{\extracolsep{5pt}}lccc} 
\\[-1.8ex]\hline 
\hline \\[-1.8ex] 
 & \multicolumn{3}{c}{\textit{Dependent variable:}} \\ 
\cline{2-4} 
\\[-1.8ex] & \multicolumn{3}{c}{$\% \Delta GDP_{t}$} \\ 
\\[-1.8ex] & \textit{Pooled} 
 & \textit{Country FE} & \textit{Country + Year FE} \\ 
\\[-1.8ex] & (1) & (2) & (3)\\ 
\hline \\[-1.8ex] 
 Civil War$_{t-1}$ & $-$7.210$^{***}$ & $-$8.711$^{***}$ & $-$8.807$^{***}$ \\ 
  & (1.383) & (1.707) & (1.713) \\ 
  & & & \\ 
 Ln(Min. Cap. Dist.)$_{t-1}$ & 1.245$^{***}$ & 1.158$^{***}$ & 1.175$^{***}$ \\ 
  & (0.247) & (0.308) & (0.309) \\ 
  & & & \\ 
 Ln(Inflation)$_{t-1}$ & $-$2.925$^{***}$ & $-$2.973$^{***}$ & $-$3.164$^{***}$ \\ 
  & (0.200) & (0.228) & (0.237) \\ 
  & & & \\ 
 Democracy$_{t-1}$ & $-$0.069$^{***}$ & 0.038 & 0.069 \\ 
  & (0.020) & (0.045) & (0.048) \\ 
  & & & \\ 
 Resource Rents/GDP$_{t-1}$ & 0.053$^{***}$ & 0.115$^{***}$ & 0.119$^{***}$ \\ 
  & (0.010) & (0.018) & (0.019) \\ 
  & & & \\ 
 World GDP Growth$_{t}$ & 0.751$^{***}$ & 0.662$^{***}$ &  \\ 
  & (0.083) & (0.081) &  \\ 
  & & & \\ 
 % Intercept & 4.017$^{***}$ & 12.731$^{***}$ &  &  \\ 
 %  & (0.145) & (0.959) &  &  \\ 
 %  & & & & \\ 
\hline \\[-1.8ex] 
Countries & 160 & 160 & 160 \\
Observations & 3,002 & 3,002 & 3,002 \\ 
% R$^{2}$ & 0.006 & 0.140 & 0.151 & 0.094 \\ 
% Adjusted R$^{2}$ & 0.006 & 0.139 & 0.142 & 0.088 \\ 
% Residual Std. Error & 6.981 (df = 4105) & 6.761 (df = 2995) &  &  \\ 
% F Statistic & 12.942$^{***}$ (df = 2; 4105) & 81.502$^{***}$ (df = 6; 2995) & 83.775$^{***}$ (df = 6; 2836) & 58.471$^{***}$ (df = 5; 2818) \\ 
\hline 
\hline \\[-1.8ex] 
\textit{Note:}  & \multicolumn{3}{r}{$^{*}$p$<$0.1; $^{**}$p$<$0.05; $^{***}$p$<$0.01} \\ 
\end{tabular} 
\end{table} 
\FloatBarrier


