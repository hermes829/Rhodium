\section*{Reviewer}

\begin{enumerate}
\item Empirical Strategy \& Sample: 
\begin{enumerate}
	\item   Empirical strategy. As I have already mentioned in my first report, the empirical strategy is still a
	major issue. I’m not convinced by the use of random fixed effects. At least, I would to like see a
	table following the road map I gave in my previous comments. Last, I do not understand the last
	argument about the weak time-variation of the distance to conflict. \\

	\item Sample. I am not sure about the accuracy of the answer to my third point. I asked for the use of a
	full sample of countries. I don’t see a problem to estimate the following equation: $GDP_{it} = \beta_{1}Conflict_{it} + \beta_{2}Distance_{it}+\ldots+\zeta_{it}$. where $Distance_{it}=0$ when $Conflict_{it}=0$ and $Distance_{it}>0$ when $Conflict_{it}=1$. This specification allows to consider the full sample, to control for conflict and to estimate the main story about the distance to events. I expect the estimate of $\beta_{2}$ to be positive. \\
\end{enumerate}

\begin{itemize}
\item \textcolor{blue}{\emph{
	The specification suggested above is one that we have now incorporated into our paper. We adopt the following specification suggested by the reviewer: $GDP_{it} = \beta_{1}Conflict_{it} + \beta_{2}Distance_{it}+\ldots+\zeta_{it}$. where $Distance_{it}=0$ when $Conflict_{it}=0$ and $Distance_{it}>0$ when $Conflict_{it}=1$. Further as suggested by the reviewer we estimate the model with fixed effects instead of random effects. The results are shown below for both our minimum city distance and minimum capital distance variables, see tables \ref{tab:cityFullPiecewiseFE} and \ref{tab:capFullPiecewiseFE}, respectively. In line with the expectation the reviewer suggested the effect of $\beta_{2}$ is positive across each specification. We have included an appendix item into the paper describing these results.
}} 
\item City models
% Piecewise mindist fixed effects model
% Table created by stargazer v.5.2 by Marek Hlavac, Harvard University. E-mail: hlavac at fas.harvard.edu
% Date and time: Thu, Nov 05, 2015 - 23:34:34
\begin{table}[!htbp] \centering 
  \caption{This table shows the results of a series of regressions estimating the effect of a conflict distance to major cities on GDP growth in which we utilize a full country year panel. The first column shows the results of our base model with controls estimated with no fixed effects, next we add fixed effects for countries, and last we incorporate both country and year fixed effects.} 
  \label{tab:cityFullPiecewiseFE} 
\begin{tabular}{@{\extracolsep{5pt}}lccc} 
\\[-1.8ex]\hline 
\hline \\[-1.8ex] 
 & \multicolumn{3}{c}{\textit{Dependent variable:}} \\ 
\cline{2-4} 
\\[-1.8ex] & \multicolumn{3}{c}{$\% \Delta GDP_{t}$} \\ 
\\[-1.8ex] & \textit{Pooled} 
 & \textit{Country FE} & \textit{Country + Year FE} \\ 
\\[-1.8ex] & (1) & (2) & (3)\\ 
\hline \\[-1.8ex] 
 Civil War$_{t-1}$ & $-$6.899$^{***}$ & $-$8.979$^{***}$ & $-$9.145$^{***}$ \\ 
  & (1.369) & (1.690) & (1.697) \\ 
  & & & \\ 
 Ln(Min. City Dist.)$_{t-1}$ & 1.267$^{***}$ & 1.281$^{***}$ & 1.312$^{***}$ \\ 
  & (0.260) & (0.322) & (0.324) \\ 
  & & & \\ 
 Ln(Inflation)$_{t-1}$ & $-$2.908$^{***}$ & $-$2.963$^{***}$ & $-$3.156$^{***}$ \\ 
  & (0.200) & (0.228) & (0.237) \\ 
  & & & \\ 
 Democracy$_{t-1}$ & $-$0.065$^{***}$ & 0.047 & 0.081$^{*}$ \\ 
  & (0.020) & (0.045) & (0.048) \\ 
  & & & \\ 
 Resource Rents/GDP$_{t-1}$ & 0.054$^{***}$ & 0.116$^{***}$ & 0.120$^{***}$ \\ 
  & (0.010) & (0.018) & (0.019) \\ 
  & & & \\ 
 World GDP Growth$_{t}$ & 0.756$^{***}$ & 0.663$^{***}$ &  \\ 
  & (0.083) & (0.081) &  \\ 
  & & & \\ 
 % Intercept & 4.017$^{***}$ & 12.582$^{***}$ &  &  \\ 
 %  & (0.145) & (0.959) &  &  \\ 
 %  & & & & \\ 
\hline \\[-1.8ex] 
Countries & 160 & 160 & 160 \\
Observations & 3,002 & 3,002 & 3,002 \\ 
% R$^{2}$ & 0.013 & 0.140 & 0.151 & 0.095 \\ 
% Adjusted R$^{2}$ & 0.013 & 0.138 & 0.143 & 0.089 \\ 
% Residual Std. Error & 7.238 (df = 2999) & 6.763 (df = 2995) &  &  \\ 
% F Statistic & 20.180$^{***}$ (df = 2; 2999) & 81.155$^{***}$ (df = 6; 2995) & 84.088$^{***}$ (df = 6; 2836) & 58.895$^{***}$ (df = 5; 2818) \\ 
\hline 
\hline \\[-1.8ex] 
\textit{Note:}  & \multicolumn{3}{r}{$^{*}$p$<$0.1; $^{**}$p$<$0.05; $^{***}$p$<$0.01} \\ 
\end{tabular} 
\end{table} 
\item Capital city models
% Piecewise capdist fixed effects model
% Table created by stargazer v.5.2 by Marek Hlavac, Harvard University. E-mail: hlavac at fas.harvard.edu
% Date and time: Thu, Nov 05, 2015 - 23:34:55
\begin{table}[!htbp] \centering 
  \caption{This table shows the results of a series of regressions estimating the effect of a conflict distance to capital cities on GDP growth in which we utilize a full country year panel. The first column shows the results of our base model with controls estimated with no fixed effects, next we add fixed effects for countries, and last we incorporate both country and year fixed effects.} 
  \label{tab:capFullPiecewiseFE} 
\begin{tabular}{@{\extracolsep{5pt}}lccc} 
\\[-1.8ex]\hline 
\hline \\[-1.8ex] 
 & \multicolumn{3}{c}{\textit{Dependent variable:}} \\ 
\cline{2-4} 
\\[-1.8ex] & \multicolumn{3}{c}{$\% \Delta GDP_{t}$} \\ 
\\[-1.8ex] & \textit{Pooled} 
 & \textit{Country FE} & \textit{Country + Year FE} \\ 
\\[-1.8ex] & (1) & (2) & (3)\\ 
\hline \\[-1.8ex] 
 Civil War$_{t-1}$ & $-$7.210$^{***}$ & $-$8.711$^{***}$ & $-$8.807$^{***}$ \\ 
  & (1.383) & (1.707) & (1.713) \\ 
  & & & \\ 
 Ln(Min. Cap. Dist.)$_{t-1}$ & 1.245$^{***}$ & 1.158$^{***}$ & 1.175$^{***}$ \\ 
  & (0.247) & (0.308) & (0.309) \\ 
  & & & \\ 
 Ln(Inflation)$_{t-1}$ & $-$2.925$^{***}$ & $-$2.973$^{***}$ & $-$3.164$^{***}$ \\ 
  & (0.200) & (0.228) & (0.237) \\ 
  & & & \\ 
 Democracy$_{t-1}$ & $-$0.069$^{***}$ & 0.038 & 0.069 \\ 
  & (0.020) & (0.045) & (0.048) \\ 
  & & & \\ 
 Resource Rents/GDP$_{t-1}$ & 0.053$^{***}$ & 0.115$^{***}$ & 0.119$^{***}$ \\ 
  & (0.010) & (0.018) & (0.019) \\ 
  & & & \\ 
 World GDP Growth$_{t}$ & 0.751$^{***}$ & 0.662$^{***}$ &  \\ 
  & (0.083) & (0.081) &  \\ 
  & & & \\ 
 % Intercept & 4.017$^{***}$ & 12.731$^{***}$ &  &  \\ 
 %  & (0.145) & (0.959) &  &  \\ 
 %  & & & & \\ 
\hline \\[-1.8ex] 
Countries & 160 & 160 & 160 \\
Observations & 3,002 & 3,002 & 3,002 \\ 
% R$^{2}$ & 0.006 & 0.140 & 0.151 & 0.094 \\ 
% Adjusted R$^{2}$ & 0.006 & 0.139 & 0.142 & 0.088 \\ 
% Residual Std. Error & 6.981 (df = 4105) & 6.761 (df = 2995) &  &  \\ 
% F Statistic & 12.942$^{***}$ (df = 2; 4105) & 81.502$^{***}$ (df = 6; 2995) & 83.775$^{***}$ (df = 6; 2836) & 58.471$^{***}$ (df = 5; 2818) \\ 
\hline 
\hline \\[-1.8ex] 
\textit{Note:}  & \multicolumn{3}{r}{$^{*}$p$<$0.1; $^{**}$p$<$0.05; $^{***}$p$<$0.01} \\ 
\end{tabular} 
\end{table} 
\end{itemize}

\newpage
\item The authors have to consider the country’s size. The distance has to be weighted by the country size. It cannot be an argument to rule out the possibility to use country fixed effects. \\

\begin{itemize}
\item \textcolor{blue}{\emph{
	We reran the analysis weighting both of our conflict distance variables by area and the results again hold. We do not include them in the paper but show the results below. 
}}
\input{distWeightArea.tex}
\end{itemize}

\item The sample includes the 1997 Asian crisis and the 2008 financial crisis. The authors control for the average GDP growth across all countries. The better way to control for time-invariant common shocks is the inclusion of year fixed effects. \\

\begin{itemize}
\item \textcolor{blue}{\emph{
	We run an alternative model using year fixed effects instead of a lagged GDP growth measure. Results are shown below are findings still hold.
}}
\input{cntryTimeFE.tex}
\end{itemize}

\item Data. Considering the raw dataset on conflict, US and Spain were in war (Figure 3). PRIO defines different nature of conflicts. I would like to see results with this distinctions. \\

\textcolor{blue}{\emph{
	We provide this distinction in 
	We reran the models per this specification and have included the results in the appendix. In both low intensity and high intensity cases we find that the distance variables remain significant and in the hypothesized direction.
}}

\item The sub-section 3.3 on the descriptive cases is a good candidate for the (online) appendix. \\

\textcolor{blue}{\emph{
	stuff
}}

\item  The authors include binary indicators for whether the country is classified as upper income by the World Bank. It is highly endogeneous to the main variable of interest. \\

\textcolor{blue}{\emph{
	Removing this variable has no effect on our analysis. We do not include this extra check in the paper but results are available upon request.
}}

\item The authors have to motivate why they decide to introduce the main variables with lags. Link to the introduction of lags, it would be interesting in the same framework to see whether it is possible \\

\textcolor{blue}{\emph{
	We don't understand this point.
}}

\item Table 2. The authors mention random fixed effects with country, year or country + year. I do not understand what they mean. \\

\textcolor{blue}{\emph{
	We have replaced this table per your suggestion. 
}}

\item Tables have to be self-contained. \\

\textcolor{blue}{\emph{
	We don't understand this point.
}}

\item The references have to actualized. \\

\textcolor{blue}{\emph{
	stuff
}}

\end{enumerate}