\section*{Reviewer}

\begin{enumerate}
\item Empirical Strategy \& Sample: \textcolor{blue}{\emph{The reviewer reiterates a number of points regarding our empirical strategy. We have bundled these comments, each shown below, into this single section so that we can address them in a holistic manner. The main concerns the reviewer raises are with the issue of estimating on a full sample versus conflict-year sample, use of random effects, and the presentation of the results.}} \\
\begin{enumerate}
	\item   Empirical strategy. As I have already mentioned in my first report, the empirical strategy is still a major issue. I’m not convinced by the use of random fixed effects. At least, I would to like see a
	table following the road map I gave in my previous comments. Last, I do not understand the last
	argument about the weak time-variation of the distance to conflict. \\

	\item Sample. I am not sure about the accuracy of the answer to my third point. I asked for the use of a
	full sample of countries. I don’t see a problem to estimate the following equation: $GDP_{it} = \beta_{1}Conflict_{it} + \beta_{2}Distance_{it}+\ldots+\zeta_{it}$. where $Distance_{it}=0$ when $Conflict_{it}=0$ and $Distance_{it}>0$ when $Conflict_{it}=1$. This specification allows to consider the full sample, to control for conflict and to estimate the main story about the distance to events. I expect the estimate of $\beta_{2}$ to be positive. \\

	\item The sample includes the 1997 Asian crisis and the 2008 financial crisis. The authors control for the average GDP growth across all countries. The better way to control for time-invariant common shocks is the inclusion of year fixed effects. \\	

	\item  The authors include binary indicators for whether the country is classified as upper income by the World Bank. It is highly endogeneous to the main variable of interest. \\	

	\item Table 2. The authors mention random fixed effects with country, year or country + year. I do not understand what they mean. \\	
\end{enumerate}

\begin{itemize}
\item \textcolor{blue}{\emph{
	We adopt the following specification suggested by the reviewer: $GDP_{it} = \beta_{1}Conflict_{it} + \beta_{2}Distance_{it}+\ldots+\zeta_{it}$. where $Distance_{it}=0$ when $Conflict_{it}=0$ and $Distance_{it}>0$ when $Conflict_{it}=1$. Additionally, as suggested by the reviewer, we estimate the model with fixed instead of random effects. The results are shown below for both our minimum city distance and minimum capital distance variables, see tables \ref{tab:cityFullPiecewiseFE} and \ref{tab:capFullPiecewiseFE}, respectively. In line with the expectation the reviewer suggested the effect of $\beta_{2}$ is positive across each specification. We have included an appendix item into the paper describing these results.
}} \\

\item \textcolor{blue}{\emph{
	To deal with the reviewer's assertion that year fixed effects are more appropriate than a average GDP growth measure, in the last column of both tables \ref{tab:cityFullPiecewiseFE} and \ref{tab:capFullPiecewiseFE} we exclude the average GDP growth measure and instead use year fixed effects. The results are still consistent with our hypotheses for both the minimum city and capital distance variables.
}} \\

\item \textcolor{blue}{\emph{
	The reviewer also raises concerns about the endogeneity of the upper income variable, we exclude it in each of the fixed effects specifications here since it is time-invariant, and as you can see the results remain consistent with our hypotheses.
}} \\

\item \textcolor{blue}{\emph{
	We had originally included results broken down by different hierarchies of effects according to earlier reviewer comments. Since those were unclear we redo the analysis and present the results below in tables \ref{tab:cityFullPiecewiseFE} and \ref{tab:capFullPiecewiseFE}. These analyses are performed on the full panel of data and therefore are missing conflict-specific covariates that are unavailable for countries not experiencing civil war. In the first column, we show the results of our baseline model without any fixed effects, next we add fixed effects for countries, and last we include country+year fixed effects.
}} \\

\end{itemize}

% Piecewise mindist fixed effects model
% Table created by stargazer v.5.2 by Marek Hlavac, Harvard University. E-mail: hlavac at fas.harvard.edu
% Date and time: Thu, Nov 05, 2015 - 23:34:34
\begin{table}[!htbp] \centering 
  \caption{This table shows the results of a series of regressions estimating the effect of a conflict distance to major cities on GDP growth in which we utilize a full country year panel. The first column shows the results of our base model with controls estimated with no fixed effects, next we add fixed effects for countries, and last we incorporate both country and year fixed effects.} 
  \label{tab:cityFullPiecewiseFE} 
\begin{tabular}{@{\extracolsep{5pt}}lccc} 
\\[-1.8ex]\hline 
\hline \\[-1.8ex] 
 & \multicolumn{3}{c}{\textit{Dependent variable:}} \\ 
\cline{2-4} 
\\[-1.8ex] & \multicolumn{3}{c}{$\% \Delta GDP_{t}$} \\ 
\\[-1.8ex] & \textit{Pooled} 
 & \textit{Country FE} & \textit{Country + Year FE} \\ 
\\[-1.8ex] & (1) & (2) & (3)\\ 
\hline \\[-1.8ex] 
 Civil War$_{t-1}$ & $-$6.899$^{***}$ & $-$8.979$^{***}$ & $-$9.145$^{***}$ \\ 
  & (1.369) & (1.690) & (1.697) \\ 
  & & & \\ 
 Ln(Min. City Dist.)$_{t-1}$ & 1.267$^{***}$ & 1.281$^{***}$ & 1.312$^{***}$ \\ 
  & (0.260) & (0.322) & (0.324) \\ 
  & & & \\ 
 Ln(Inflation)$_{t-1}$ & $-$2.908$^{***}$ & $-$2.963$^{***}$ & $-$3.156$^{***}$ \\ 
  & (0.200) & (0.228) & (0.237) \\ 
  & & & \\ 
 Democracy$_{t-1}$ & $-$0.065$^{***}$ & 0.047 & 0.081$^{*}$ \\ 
  & (0.020) & (0.045) & (0.048) \\ 
  & & & \\ 
 Resource Rents/GDP$_{t-1}$ & 0.054$^{***}$ & 0.116$^{***}$ & 0.120$^{***}$ \\ 
  & (0.010) & (0.018) & (0.019) \\ 
  & & & \\ 
 World GDP Growth$_{t}$ & 0.756$^{***}$ & 0.663$^{***}$ &  \\ 
  & (0.083) & (0.081) &  \\ 
  & & & \\ 
 % Intercept & 4.017$^{***}$ & 12.582$^{***}$ &  &  \\ 
 %  & (0.145) & (0.959) &  &  \\ 
 %  & & & & \\ 
\hline \\[-1.8ex] 
Countries & 160 & 160 & 160 \\
Observations & 3,002 & 3,002 & 3,002 \\ 
% R$^{2}$ & 0.013 & 0.140 & 0.151 & 0.095 \\ 
% Adjusted R$^{2}$ & 0.013 & 0.138 & 0.143 & 0.089 \\ 
% Residual Std. Error & 7.238 (df = 2999) & 6.763 (df = 2995) &  &  \\ 
% F Statistic & 20.180$^{***}$ (df = 2; 2999) & 81.155$^{***}$ (df = 6; 2995) & 84.088$^{***}$ (df = 6; 2836) & 58.895$^{***}$ (df = 5; 2818) \\ 
\hline 
\hline \\[-1.8ex] 
\textit{Note:}  & \multicolumn{3}{r}{$^{*}$p$<$0.1; $^{**}$p$<$0.05; $^{***}$p$<$0.01} \\ 
\end{tabular} 
\end{table} 
% Piecewise capdist fixed effects model
% Table created by stargazer v.5.2 by Marek Hlavac, Harvard University. E-mail: hlavac at fas.harvard.edu
% Date and time: Thu, Nov 05, 2015 - 23:34:55
\begin{table}[!htbp] \centering 
  \caption{This table shows the results of a series of regressions estimating the effect of a conflict distance to capital cities on GDP growth in which we utilize a full country year panel. The first column shows the results of our base model with controls estimated with no fixed effects, next we add fixed effects for countries, and last we incorporate both country and year fixed effects.} 
  \label{tab:capFullPiecewiseFE} 
\begin{tabular}{@{\extracolsep{5pt}}lccc} 
\\[-1.8ex]\hline 
\hline \\[-1.8ex] 
 & \multicolumn{3}{c}{\textit{Dependent variable:}} \\ 
\cline{2-4} 
\\[-1.8ex] & \multicolumn{3}{c}{$\% \Delta GDP_{t}$} \\ 
\\[-1.8ex] & \textit{Pooled} 
 & \textit{Country FE} & \textit{Country + Year FE} \\ 
\\[-1.8ex] & (1) & (2) & (3)\\ 
\hline \\[-1.8ex] 
 Civil War$_{t-1}$ & $-$7.210$^{***}$ & $-$8.711$^{***}$ & $-$8.807$^{***}$ \\ 
  & (1.383) & (1.707) & (1.713) \\ 
  & & & \\ 
 Ln(Min. Cap. Dist.)$_{t-1}$ & 1.245$^{***}$ & 1.158$^{***}$ & 1.175$^{***}$ \\ 
  & (0.247) & (0.308) & (0.309) \\ 
  & & & \\ 
 Ln(Inflation)$_{t-1}$ & $-$2.925$^{***}$ & $-$2.973$^{***}$ & $-$3.164$^{***}$ \\ 
  & (0.200) & (0.228) & (0.237) \\ 
  & & & \\ 
 Democracy$_{t-1}$ & $-$0.069$^{***}$ & 0.038 & 0.069 \\ 
  & (0.020) & (0.045) & (0.048) \\ 
  & & & \\ 
 Resource Rents/GDP$_{t-1}$ & 0.053$^{***}$ & 0.115$^{***}$ & 0.119$^{***}$ \\ 
  & (0.010) & (0.018) & (0.019) \\ 
  & & & \\ 
 World GDP Growth$_{t}$ & 0.751$^{***}$ & 0.662$^{***}$ &  \\ 
  & (0.083) & (0.081) &  \\ 
  & & & \\ 
 % Intercept & 4.017$^{***}$ & 12.731$^{***}$ &  &  \\ 
 %  & (0.145) & (0.959) &  &  \\ 
 %  & & & & \\ 
\hline \\[-1.8ex] 
Countries & 160 & 160 & 160 \\
Observations & 3,002 & 3,002 & 3,002 \\ 
% R$^{2}$ & 0.006 & 0.140 & 0.151 & 0.094 \\ 
% Adjusted R$^{2}$ & 0.006 & 0.139 & 0.142 & 0.088 \\ 
% Residual Std. Error & 6.981 (df = 4105) & 6.761 (df = 2995) &  &  \\ 
% F Statistic & 12.942$^{***}$ (df = 2; 4105) & 81.502$^{***}$ (df = 6; 2995) & 83.775$^{***}$ (df = 6; 2836) & 58.471$^{***}$ (df = 5; 2818) \\ 
\hline 
\hline \\[-1.8ex] 
\textit{Note:}  & \multicolumn{3}{r}{$^{*}$p$<$0.1; $^{**}$p$<$0.05; $^{***}$p$<$0.01} \\ 
\end{tabular} 
\end{table} 
\FloatBarrier

\newpage
\item The authors have to consider the country’s size. The distance has to be weighted by the country size. It cannot be an argument to rule out the possibility to use country fixed effects. \\

\begin{itemize}
\item \textcolor{blue}{\emph{
	We reran the analysis weighting both of our conflict distance variables by area and the results remain consistent with our hypotheses. We do not include this set of findings in the appendix of our paper but would be happy to do so.
}} \\
\input{distWeightArea.tex}
\end{itemize}
\FloatBarrier

\newpage
\item Data. Considering the raw dataset on conflict, US and Spain were in war (Figure 3). PRIO defines different nature of conflicts. I would like to see results with this distinctions. \\

\textcolor{blue}{\emph{
	The reviewer comments from the first round had also suggested breaking down the results by different natures of conflict. To account for this, we re-did our primary models estimating the effect of distance on growth, but restricting to the civil conflicts coded as wars and then a separate model for civil conflicts coded as low intensity events. In both low intensity and high intensity cases we find that the conflict distance variables remain significant and in the expected direction, but the $\beta$ estimate of our distance variables is noticeably higher when using high intensity versus low intensity civil conflict cases. The results are presented in Table~\ref{tab:modHiLoIntensity} below and are included as an appendix item.
}} \\
\begin{table}[!htbp] \centering 
  \caption{Random effects regressions by PRIO intensity. } 
  \label{tab:modHiLoIntensity} 
\footnotesize{
\begin{tabular}{@{\extracolsep{5pt}}lcccc} 
\\[-1.8ex]\hline 
\hline \\[-1.8ex] 
 & \multicolumn{4}{c}{\textit{Dependent variable:}} \\ 
\cline{2-5} 
\\[-1.8ex] & \multicolumn{4}{c}{$\% \Delta GDP_{t}$} \\ 
\\[-1.8ex] & (Low Intensity) & (High Intensity) & (Low Intensity) & (High Intensity)\\ 
\hline \\[-1.8ex] 
 Ln(Min. City Dist.)$_{t-1}$ & 1.163$^{***}$ & 2.281$^{**}$ &  &  \\ 
  & (0.409) & (1.130) &  &  \\ 
  & & & & \\ 
 Ln(Min. Cap. Dist.)$_{t-1}$ &  &  & 1.009$^{***}$ & 2.884$^{***}$ \\ 
  &  &  & (0.385) & (1.104) \\ 
  & & & & \\ 
 Duration$_{t-1}$ & 0.151$^{***}$ & 0.227$^{**}$ & 0.153$^{***}$ & 0.204$^{**}$ \\ 
  & (0.035) & (0.091) & (0.035) & (0.090) \\ 
  & & & & \\ 
 Area$_{t-1}$ & $-$3.794$^{***}$ & $-$8.995$^{***}$ & $-$3.603$^{***}$ & $-$7.606$^{***}$ \\ 
  & (1.345) & (2.636) & (1.366) & (2.703) \\ 
  & & & & \\ 
 Number of conflicts$_{t-1}$ & 1.367$^{**}$ & 1.262 & 1.332$^{**}$ & 1.406 \\ 
  & (0.573) & (3.599) & (0.573) & (3.556) \\ 
  & & & & \\ 
 Upper Income & 2.176 & $-$1.637 & 1.741 & $-$0.390 \\ 
  & (2.342) & (9.430) & (2.300) & (9.316) \\ 
  & & & & \\ 
 Ln(Inflation)$_{t-1}$ & $-$2.020$^{***}$ & $-$2.984$^{***}$ & $-$2.087$^{***}$ & $-$3.030$^{***}$ \\ 
  & (0.499) & (0.727) & (0.497) & (0.713) \\ 
  & & & & \\ 
 Democracy$_{t-1}$ & $-$0.051 & 0.117 & $-$0.073 & 0.118 \\ 
  & (0.089) & (0.214) & (0.089) & (0.211) \\ 
  & & & & \\ 
 Resource Rents/GDP$_{t-1}$ & 0.106$^{***}$ & $-$0.034 & 0.107$^{***}$ & $-$0.052 \\ 
  & (0.036) & (0.067) & (0.036) & (0.067) \\ 
  & & & & \\ 
 World GDP Growth$_{t}$ & 0.560$^{*}$ & 0.461 & 0.546$^{*}$ & 0.422 \\ 
  & (0.299) & (0.482) & (0.300) & (0.476) \\ 
  & & & & \\ 
 Intercept & $-$1.315 & $-$3.701 & $-$0.387 & $-$7.567 \\ 
  & (3.504) & (9.592) & (3.395) & (9.453) \\ 
  & & & & \\ 
\hline \\[-1.8ex] 
Countries & 66 & 30 & 66 & 30 \\ 
Observations & 403 & 131 & 403 & 131 \\ 
\hline 
\hline \\[-1.8ex] 
\textit{Note:}  & \multicolumn{4}{l}{$^{*}$p$<$0.1; $^{**}$p$<$0.05; $^{***}$p$<$0.01} \\ 
\end{tabular}
} 
\end{table} 
\FloatBarrier

\item The authors have to motivate why they decide to introduce the main variables with lags. Link to the introduction of lags, it would be interesting in the same framework to see whether it is possible \\

\textcolor{blue}{\emph{
	Lagged independent variables are common in the econometrics literature. In our case, lagging the civil war-related terms guarantees a temporal ordering of conflict followed by the GDP growth estimate so we can discount the possibility that our measures of conflict reflect events that happened near or at the end of the measurement period of the economic variables. If our assumption is incorrect, we anticipate that this will negatively impact our findings by moderating the effect of conflict on GDP growth. Indeed, previous work accords with this notion of a temporal lag of civil war's effect on economic performance. For example, Collier (1999) estimates the effect of ``months since civil war'' on economic performance in addition to a contemporaneous effect of civil war on economic performance. The contemporaneous effect, in that case, is the result of a decade average of both variables that is not applicable to our country-year panel. The ``months since civil war'' is a generalization of our one year lagged parameterization.
}} \\

\item The sub-section 3.3 on the descriptive cases is a good candidate for the (online) appendix. \\

\textcolor{blue}{\emph{
	For now we would prefer to keep those descriptive cases in as they help the reader to understand our theory in the context of a few concrete examples. However, given word limits we would be happy to move those items to an appendix as well. 
}} \\

\item Tables have to be self-contained. \\

\textcolor{blue}{\emph{
	We have added additional information in the captions for each of the tables.
}} \\

\item The references have to actualized. \\

\textcolor{blue}{\emph{
	We have checked to make sure that all references have been made available.
}} \\

\end{enumerate}