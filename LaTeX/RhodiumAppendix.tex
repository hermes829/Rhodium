
\newpage
\section{Appendix}
\label{appendix}

\subsection{Full Panel Fixed Effects Analysis}

Here we broaden the random effect conflict-year analysis presented in the paper to a country-year analysis in which we employ a fixed effects framework. Specifically, we estimate the following equation: $\% \delta GDP_{i,t} = \beta_{1}Conflict_{i,t-1} + \beta_{2}Distance_{i,t-1}+\ldots+\zeta_{i,t-1}$, where $Distance_{i,t-1}=0$ when $Conflict_{i,t-1}=0$ and $Distance_{i,t-1}>0$ when $Conflict_{i,t-1}=1$. The $\zeta_{i,t-1}$ term represents our bundle of control variables. This specification provides one way to consider the full sample, control for the presence of armed conflict, and test our key hypothesis regarding the role of conflict proximity to major urban centers. In this specification, we expect the estimate of $\beta_{2}$ to be positive.

The results are presented in Tables~\ref{tab:cityFullPiecewiseFE} and \ref{tab:capFullPiecewiseFE}. In the first column of these tables, we run a pooled country-year analysis in which we include our key variable of interest, distance to conflict, and controls. In the second column, we incorporate country fixed effects and in the last we utilize both country and year fixed effects. When adding year fixed effects we remove our global average GDP growth measure. Since these analyses are performed on the full panel of data we have to exclude the conflict-specific covariates from the PRIO dataset as they are undefined for countries not experiencing civil war. 

When utilizing this alternative approach we still find strong support for the hypotheses that countries experiencing conflict more proximate to capital or major cities are more likely to see significant reductions in GDP growth than conflicts that are farther away from these economically vital centers.

% Piecewise mindist fixed effects model
% Table created by stargazer v.5.2 by Marek Hlavac, Harvard University. E-mail: hlavac at fas.harvard.edu
% Date and time: Thu, Nov 05, 2015 - 23:34:34
\begin{table}[!htbp] \centering 
  \caption{This table shows the results of a series of regressions estimating the effect of a conflict distance to major cities on GDP growth in which we utilize a full country year panel. The first column shows the results of our base model with controls estimated with no fixed effects, next we add fixed effects for countries, and last we incorporate both country and year fixed effects.} 
  \label{tab:cityFullPiecewiseFE} 
\begin{tabular}{@{\extracolsep{5pt}}lccc} 
\\[-1.8ex]\hline 
\hline \\[-1.8ex] 
 & \multicolumn{3}{c}{\textit{Dependent variable:}} \\ 
\cline{2-4} 
\\[-1.8ex] & \multicolumn{3}{c}{$\% \Delta GDP_{t}$} \\ 
\\[-1.8ex] & \textit{Pooled} 
 & \textit{Country FE} & \textit{Country + Year FE} \\ 
\\[-1.8ex] & (1) & (2) & (3)\\ 
\hline \\[-1.8ex] 
 Civil War$_{t-1}$ & $-$6.899$^{***}$ & $-$8.979$^{***}$ & $-$9.145$^{***}$ \\ 
  & (1.369) & (1.690) & (1.697) \\ 
  & & & \\ 
 Ln(Min. City Dist.)$_{t-1}$ & 1.267$^{***}$ & 1.281$^{***}$ & 1.312$^{***}$ \\ 
  & (0.260) & (0.322) & (0.324) \\ 
  & & & \\ 
 Ln(Inflation)$_{t-1}$ & $-$2.908$^{***}$ & $-$2.963$^{***}$ & $-$3.156$^{***}$ \\ 
  & (0.200) & (0.228) & (0.237) \\ 
  & & & \\ 
 Democracy$_{t-1}$ & $-$0.065$^{***}$ & 0.047 & 0.081$^{*}$ \\ 
  & (0.020) & (0.045) & (0.048) \\ 
  & & & \\ 
 Resource Rents/GDP$_{t-1}$ & 0.054$^{***}$ & 0.116$^{***}$ & 0.120$^{***}$ \\ 
  & (0.010) & (0.018) & (0.019) \\ 
  & & & \\ 
 World GDP Growth$_{t}$ & 0.756$^{***}$ & 0.663$^{***}$ &  \\ 
  & (0.083) & (0.081) &  \\ 
  & & & \\ 
 % Intercept & 4.017$^{***}$ & 12.582$^{***}$ &  &  \\ 
 %  & (0.145) & (0.959) &  &  \\ 
 %  & & & & \\ 
\hline \\[-1.8ex] 
Countries & 160 & 160 & 160 \\
Observations & 3,002 & 3,002 & 3,002 \\ 
% R$^{2}$ & 0.013 & 0.140 & 0.151 & 0.095 \\ 
% Adjusted R$^{2}$ & 0.013 & 0.138 & 0.143 & 0.089 \\ 
% Residual Std. Error & 7.238 (df = 2999) & 6.763 (df = 2995) &  &  \\ 
% F Statistic & 20.180$^{***}$ (df = 2; 2999) & 81.155$^{***}$ (df = 6; 2995) & 84.088$^{***}$ (df = 6; 2836) & 58.895$^{***}$ (df = 5; 2818) \\ 
\hline 
\hline \\[-1.8ex] 
\textit{Note:}  & \multicolumn{3}{r}{$^{*}$p$<$0.1; $^{**}$p$<$0.05; $^{***}$p$<$0.01} \\ 
\end{tabular} 
\end{table} 
\FloatBarrier

% Piecewise capdist fixed effects model
% Table created by stargazer v.5.2 by Marek Hlavac, Harvard University. E-mail: hlavac at fas.harvard.edu
% Date and time: Thu, Nov 05, 2015 - 23:34:55
\begin{table}[!htbp] \centering 
  \caption{This table shows the results of a series of regressions estimating the effect of a conflict distance to capital cities on GDP growth in which we utilize a full country year panel. The first column shows the results of our base model with controls estimated with no fixed effects, next we add fixed effects for countries, and last we incorporate both country and year fixed effects.} 
  \label{tab:capFullPiecewiseFE} 
\begin{tabular}{@{\extracolsep{5pt}}lccc} 
\\[-1.8ex]\hline 
\hline \\[-1.8ex] 
 & \multicolumn{3}{c}{\textit{Dependent variable:}} \\ 
\cline{2-4} 
\\[-1.8ex] & \multicolumn{3}{c}{$\% \Delta GDP_{t}$} \\ 
\\[-1.8ex] & \textit{Pooled} 
 & \textit{Country FE} & \textit{Country + Year FE} \\ 
\\[-1.8ex] & (1) & (2) & (3)\\ 
\hline \\[-1.8ex] 
 Civil War$_{t-1}$ & $-$7.210$^{***}$ & $-$8.711$^{***}$ & $-$8.807$^{***}$ \\ 
  & (1.383) & (1.707) & (1.713) \\ 
  & & & \\ 
 Ln(Min. Cap. Dist.)$_{t-1}$ & 1.245$^{***}$ & 1.158$^{***}$ & 1.175$^{***}$ \\ 
  & (0.247) & (0.308) & (0.309) \\ 
  & & & \\ 
 Ln(Inflation)$_{t-1}$ & $-$2.925$^{***}$ & $-$2.973$^{***}$ & $-$3.164$^{***}$ \\ 
  & (0.200) & (0.228) & (0.237) \\ 
  & & & \\ 
 Democracy$_{t-1}$ & $-$0.069$^{***}$ & 0.038 & 0.069 \\ 
  & (0.020) & (0.045) & (0.048) \\ 
  & & & \\ 
 Resource Rents/GDP$_{t-1}$ & 0.053$^{***}$ & 0.115$^{***}$ & 0.119$^{***}$ \\ 
  & (0.010) & (0.018) & (0.019) \\ 
  & & & \\ 
 World GDP Growth$_{t}$ & 0.751$^{***}$ & 0.662$^{***}$ &  \\ 
  & (0.083) & (0.081) &  \\ 
  & & & \\ 
 % Intercept & 4.017$^{***}$ & 12.731$^{***}$ &  &  \\ 
 %  & (0.145) & (0.959) &  &  \\ 
 %  & & & & \\ 
\hline \\[-1.8ex] 
Countries & 160 & 160 & 160 \\
Observations & 3,002 & 3,002 & 3,002 \\ 
% R$^{2}$ & 0.006 & 0.140 & 0.151 & 0.094 \\ 
% Adjusted R$^{2}$ & 0.006 & 0.139 & 0.142 & 0.088 \\ 
% Residual Std. Error & 6.981 (df = 4105) & 6.761 (df = 2995) &  &  \\ 
% F Statistic & 12.942$^{***}$ (df = 2; 4105) & 81.502$^{***}$ (df = 6; 2995) & 83.775$^{***}$ (df = 6; 2836) & 58.471$^{***}$ (df = 5; 2818) \\ 
\hline 
\hline \\[-1.8ex] 
\textit{Note:}  & \multicolumn{3}{r}{$^{*}$p$<$0.1; $^{**}$p$<$0.05; $^{***}$p$<$0.01} \\ 
\end{tabular} 
\end{table} 
\FloatBarrier

\newpage
\subsection{ACLED Analysis}
\label{acled}

The Armed Conflict Location and Event Dataset provides an alternative source of information on the subnational spatial distribution of armed conflict \citep{raleigh:linke:etal:2010}. This dataset is, at the time of writing, limited to Africa and therefore was not selected for the primary analysis presented in the text. It does, however, offer us a valuable opportunity to validate our results. Here, we have replicated the primary model described in Section~\ref{empirics}.

In order to match our existing data structure, it was necessary to aggregate the ACLED data to the country-year level. We did this by first subsetting ACLED to the years 1989-2008 and then selecting only conflict sites with at least 25 fatalities in each given year. The 25 fatalities threshold is intended to mirror the PRIO coding criteria and to prevent very low-fatality events from biasing our estimates of conflict location toward high-population areas. Covariates created with PRIO but unavailable in ACLED are omitted. The analysis procedure then continues as described in Section~\ref{empirics}: the minimum distance measure is calculated as the natural logarithm of the average distance in kilometers from any conflict site to the nearest major city (or capital). The results are presented in Table~\ref{tab:acledRegResults} below. 

When utilizing this alternative dataset we again find significant support for the argument that conflicts more proximate to urban centers have a greater adverse effect on economic growth than those farther away. Additionally, if we switched to a fixed effects framework, such as the one described in the previous appendix section, we still find significant support for the hypotheses laid out in our paper.

\begin{table}[!htbp] \centering 
  \caption{Random effects regression using ACLED.} 
  \label{tab:acledRegResults} 
\footnotesize{
\begin{tabular}{@{\extracolsep{5pt}}lcc} 
\\[-1.8ex]\hline 
\hline \\[-1.8ex] 
 & \multicolumn{2}{c}{\textit{Dependent variable:}} \\ 
\cline{2-3} 
\\[-1.8ex] & \multicolumn{2}{c}{$\% \Delta GDP_{t}$} \\ 
\\[-1.8ex] & (1) & (2)\\ 
\hline \\[-1.8ex] 
 Ln(Min. City Dist.)$_{t-1}$ & 0.585$^{**}$ &  \\ 
  & (0.254) &  \\ 
  & & \\ 
 Ln(Min. Cap. Dist.)$_{t-1}$ &  & 0.660$^{**}$ \\ 
  &  & (0.259) \\ 
  & & \\ 
 Number of conflicts$_{t-1}$ & 9.091$^{***}$ & 9.291$^{***}$ \\ 
  & (1.834) & (1.832) \\ 
  & & \\ 
 Ln(Inflation)$_{t-1}$ & 1.167 & 0.921 \\ 
  & (0.851) & (0.815) \\ 
  & & \\ 
 Democracy$_{t-1}$ & 0.147 & 0.116 \\ 
  & (0.217) & (0.217) \\ 
  & & \\ 
 Resource Rents/GDP$_{t-1}$ & $-$0.0001 & 0.004 \\ 
  & (0.048) & (0.048) \\ 
  & & \\ 
 World GDP Growth$_{t}$ & 1.034$^{**}$ & 0.877$^{**}$ \\ 
  & (0.442) & (0.439) \\ 
  & & \\ 
 Intercept & $-$19.308$^{***}$ & $-$18.398$^{***}$ \\ 
  & (5.477) & (5.252) \\ 
  & & \\ 
\hline \\[-1.8ex] 
Countries & 22 & 22  \\ 
Observations & 101 & 101 \\ 
\hline 
\hline \\[-1.8ex] 
\textit{Note:}  & \multicolumn{1}{l}{$^{*}$p$<$0.1; $^{**}$p$<$0.05; $^{***}$p$<$0.01} \\ 
\end{tabular} 
}
\end{table} 
\FloatBarrier
\newpage

\subsection{Estimating Models Separately for High and Low Intensity Conflicts}

Here instead of treating conflict intensity as a control, we re-do our primary random effect regression models estimating the effect of distance on growth, but restricting to the civil conflicts coded as wars and then a separate model for civil conflicts coded as low intensity events. In both low intensity and high intensity cases we find that the conflict distance variables remain significant and in the expected direction, and the $\beta$ estimate of our distance variables is noticeably higher when using high intensity versus low intensity civil conflict cases. The results are presented in Table~\ref{tab:modHiLoIntensity} below.

\begin{table}[!htbp] \centering 
  \caption{Random effects regressions by PRIO intensity. } 
  \label{tab:modHiLoIntensity} 
\footnotesize{
\begin{tabular}{@{\extracolsep{5pt}}lcccc} 
\\[-1.8ex]\hline 
\hline \\[-1.8ex] 
 & \multicolumn{4}{c}{\textit{Dependent variable:}} \\ 
\cline{2-5} 
\\[-1.8ex] & \multicolumn{4}{c}{$\% \Delta GDP_{t}$} \\ 
\\[-1.8ex] & (Low Intensity) & (High Intensity) & (Low Intensity) & (High Intensity)\\ 
\hline \\[-1.8ex] 
 Ln(Min. City Dist.)$_{t-1}$ & 1.163$^{***}$ & 2.281$^{**}$ &  &  \\ 
  & (0.409) & (1.130) &  &  \\ 
  & & & & \\ 
 Ln(Min. Cap. Dist.)$_{t-1}$ &  &  & 1.009$^{***}$ & 2.884$^{***}$ \\ 
  &  &  & (0.385) & (1.104) \\ 
  & & & & \\ 
 Duration$_{t-1}$ & 0.151$^{***}$ & 0.227$^{**}$ & 0.153$^{***}$ & 0.204$^{**}$ \\ 
  & (0.035) & (0.091) & (0.035) & (0.090) \\ 
  & & & & \\ 
 Area$_{t-1}$ & $-$3.794$^{***}$ & $-$8.995$^{***}$ & $-$3.603$^{***}$ & $-$7.606$^{***}$ \\ 
  & (1.345) & (2.636) & (1.366) & (2.703) \\ 
  & & & & \\ 
 Number of conflicts$_{t-1}$ & 1.367$^{**}$ & 1.262 & 1.332$^{**}$ & 1.406 \\ 
  & (0.573) & (3.599) & (0.573) & (3.556) \\ 
  & & & & \\ 
 Upper Income & 2.176 & $-$1.637 & 1.741 & $-$0.390 \\ 
  & (2.342) & (9.430) & (2.300) & (9.316) \\ 
  & & & & \\ 
 Ln(Inflation)$_{t-1}$ & $-$2.020$^{***}$ & $-$2.984$^{***}$ & $-$2.087$^{***}$ & $-$3.030$^{***}$ \\ 
  & (0.499) & (0.727) & (0.497) & (0.713) \\ 
  & & & & \\ 
 Democracy$_{t-1}$ & $-$0.051 & 0.117 & $-$0.073 & 0.118 \\ 
  & (0.089) & (0.214) & (0.089) & (0.211) \\ 
  & & & & \\ 
 Resource Rents/GDP$_{t-1}$ & 0.106$^{***}$ & $-$0.034 & 0.107$^{***}$ & $-$0.052 \\ 
  & (0.036) & (0.067) & (0.036) & (0.067) \\ 
  & & & & \\ 
 World GDP Growth$_{t}$ & 0.560$^{*}$ & 0.461 & 0.546$^{*}$ & 0.422 \\ 
  & (0.299) & (0.482) & (0.300) & (0.476) \\ 
  & & & & \\ 
 Intercept & $-$1.315 & $-$3.701 & $-$0.387 & $-$7.567 \\ 
  & (3.504) & (9.592) & (3.395) & (9.453) \\ 
  & & & & \\ 
\hline \\[-1.8ex] 
Countries & 66 & 30 & 66 & 30 \\ 
Observations & 403 & 131 & 403 & 131 \\ 
\hline 
\hline \\[-1.8ex] 
\textit{Note:}  & \multicolumn{4}{l}{$^{*}$p$<$0.1; $^{**}$p$<$0.05; $^{***}$p$<$0.01} \\ 
\end{tabular}
} 
\end{table} 
\FloatBarrier
\newpage

\subsection{Baseline Effect of Civil Conflict on Growth}

In Table \ref{tab:baseline} below, we show the fixed effects regression results for the model in which we utilize a full country year panel in order to define a baseline effect of civil war on economic growth. 

% Table created by stargazer v.5.1 by Marek Hlavac, Harvard University. E-mail: hlavac at fas.harvard.edu
% Date and time: Tue, Aug 25, 2015 - 15:53:07
\begin{table}[!htbp] \centering 
  \caption{This table shows the results of a country fixed effects regression in which we are utilizing a full country year dataset. } 
  \label{tab:baseline} 
\footnotesize{
\begin{tabular}{@{\extracolsep{5pt}}lc} 
\\[-1.8ex]\hline 
\hline \\[-1.8ex] 
 & \multicolumn{1}{c}{\textit{Dependent variable:}} \\ 
\cline{2-2} 
\\[-1.8ex] & $\% \Delta GDP_{t}$ \\ 
\hline \\[-1.8ex] 
 Civil War$_{t-1}$ & $-$2.568$^{***}$ \\ 
  & (0.502) \\ 
  & \\ 
 Ln(Inflation)$_{t-1}$ & $-$3.040$^{***}$ \\ 
  & (0.228) \\ 
  & \\ 
 Democracy$_{t-1}$ & 0.043 \\ 
  & (0.045) \\ 
  & \\ 
 Resource Rents/GDP$_{t-1}$ & 0.115$^{***}$ \\ 
  & (0.018) \\ 
  & \\ 
 World GDP Growth$_{t}$ & 0.673$^{***}$ \\ 
  & (0.081) \\ 
  & \\ 
\hline \\[-1.8ex] 
Countries & 160 \\
Observations & 3,002 \\ 
\hline 
\hline \\[-1.8ex] 
\textit{Note:}  & \multicolumn{1}{l}{$^{*}$p$<$0.1; $^{**}$p$<$0.05; $^{***}$p$<$0.01} \\ 
\end{tabular} 
}
\end{table} 
\FloatBarrier
\newpage

\subsection{Conflict Distance Models in Tabular Format}

Here we present the results of our models measuring the effect of conflict distance in a tabular format, results for distance from the capital city are shown in the first column of Table \ref{tab:capminDistMod} and for any major city in the second column.  

\begin{table}[!htbp] \centering 
  \caption{Tabular representation of coefficient plots shown in Figures 4a and 4b. }
  \label{tab:capminDistMod} 
\footnotesize{
\begin{tabular}{@{\extracolsep{5pt}}lcc} 
\\[-1.8ex]\hline 
\hline \\[-1.8ex] 
 & \multicolumn{2}{c}{\textit{Dependent variable:}} \\ 
\cline{2-3} 
\\[-1.8ex] & \multicolumn{2}{c}{$\% \Delta GDP_{t}$} \\ 
\\[-1.8ex] & (1) & (2) \\ 
\hline \\[-1.8ex] 
 Ln(Min. Cap. Dist.)$_{t-1}$ & 0.994$^{**}$ &  \\ 
  & (0.394) &  \\ 
  & & \\ 
 Ln(Min. City Dist.)$_{t-1}$ &  & 1.174$^{***}$ \\ 
  &  & (0.410) \\ 
  & & \\ 
 Intensity$_{t-1}$ & $-$1.146 & $-$1.190 \\ 
  & (0.971) & (0.970) \\ 
  & & \\ 
 Duration$_{t-1}$ & 0.144$^{***}$ & 0.143$^{***}$ \\ 
  & (0.037) & (0.037) \\ 
  & & \\ 
 Area$_{t-1}$ & $-$4.704$^{***}$ & $-$4.853$^{***}$ \\ 
  & (1.295) & (1.283) \\ 
  & & \\ 
 Number of conflicts$_{t-1}$ & 1.431$^{**}$ & 1.444$^{**}$ \\ 
  & (0.617) & (0.620) \\ 
  & & \\ 
 Upper Income & 0.696 & 1.051 \\ 
  & (2.691) & (2.720) \\ 
  & & \\ 
 Ln(Inflation)$_{t-1}$ & $-$2.639$^{***}$ & $-$2.564$^{***}$ \\ 
  & (0.469) & (0.471) \\ 
  & & \\ 
 Democracy$_{t-1}$ & $-$0.024 & $-$0.003 \\ 
  & (0.090) & (0.091) \\ 
  & & \\ 
 Resource Rents/GDP$_{t-1}$ & 0.072$^{**}$ & 0.072$^{**}$ \\ 
  & (0.036) & (0.036) \\ 
  & & \\ 
 World GDP Growth$_{t}$ & 0.543$^{**}$ & 0.554$^{**}$ \\ 
  & (0.272) & (0.271) \\ 
  & & \\ 
 Constant & 2.453 & 1.360 \\ 
  & (3.349) & (3.438) \\ 
  & & \\ 
\hline \\[-1.8ex] 
Countries & 69 & 69 \\
Observations & 505 & 505 \\ 
\hline 
\hline \\[-1.8ex] 
\textit{Note:}  & \multicolumn{2}{r}{$^{*}$p$<$0.1; $^{**}$p$<$0.05; $^{***}$p$<$0.01} \\ 
\end{tabular}
} 
\end{table} 

\FloatBarrier
\newpage

