% Within the conflict literature there has been much disagreement about the relationship between civil wars, natural resources, and state economic performance. We find that this disagreement results from not accounting for the spatial disaggregation of conflict events within a country.  Other recent work has suggested that the location of civil conflicts within a state can lead to disparate outcomes in terms of conflict duration, state response, and the probability of rebel victory.  Drawing on these two literatures, we develop a theory to explain the economic consequences of civil conflict.

% Our theoretical model states that the economic impacts of civil conflict is contingent on the conflict's location relative to major economic and labor resources within a state.

% We use subnational data on the spatial distribution of conflict, resources, and infrastructure to test the long-term impact of domestic conflict on state economic performance. To estimate the spatial distribution of conflict we use data from the PRIO Armed Conflict Location and Event Data and supplement this source with GDELT in more recent years. We combine the conflict location data with geospatial data on economic centers, natural resource locations, and infrastructure grids within the country to generate country-level spatial variables that approximate how far each conflict is from centers of interest for the government. We then use a hierarchical Bayesian model to estimate the effect of our spatially constructed variables on economic performance. By doing so, we are able to resolve some of the tensions in the literature regarding the relationship between economic performance and civil wars.

% Version submitted to MPSA:
There has been much disagreement about the relationship between civil wars and state economic performance. While civil war is often associated with poor economic performance, some states have managed robust growth despite periods of domestic armed conflict. We find this disagreement results from not accounting for the spatial distribution of conflict within a country. A robust literature in economics stresses the role major cities play in economic growth. We hypothesize that the economic impact of civil conflict is contingent on the conflict's location relative to major urban centers within a state. We use subnational data on the location of conflict relative to urban areas to test the impact of domestic conflict on annual GDP growth. In doing so, we bridge the economic development literature on the importance of cities with extant literature on the effect of armed conflict to provide a novel explanation for the paradox of high macroeconomic growth in conflict ridden countries. 