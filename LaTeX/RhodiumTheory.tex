\section{Theory}
\label{theory}

% thesis: conflicts only significantly dampens economic performance if they are a threat to major population centers. conflicts isolated in sparsely populated territories of the country have little to no effect on the whole. point to the imai weinstein paper as an example of this. 

% I like the last idea where we say that conflict impedes trade between cities. 

We suspect that substate factors will determine the economic impacts of civil conflict. In particular, citizens must be able and willing to participate in commerce. When major population centers are threatened by violence, residents will be less likely to engage in economically productive activities. Violence near major population centers not only threatens residents directly, but impedes business by threatening trade between the population center and other cities or rural areas. 

% Do we even need to bring up Imai and Weinstein here? I dont think it helps us if people keep contrasting our approach with theirs as they read our paper.

While this hypothesis may resemble those of \cite{imai:weinstein:2000}, ours differs in the hypothesized mechanism through which conflict affects economic performance. While we do not disagree that the spread of a conflict could impact state economic prospects, we argue that conflict area is not necessary for adverse economic performance. Conflict area is only one possible proxy for overall destructiveness. However, conflicts with smaller spatial areas can be similarly disruptive if they are centered near (and impede access to) the urban centers described above. In fact, we anticipate measures of proximity rather than spread to be more appropriate to test the hypothesis that conflict obstructs vital economic activity.

\cite{glaeser:shapiro:2002} discuss the historic role of cities in warfare and their current role with respect to terrorism. Cities, they argue, once provided safe haven in the midst of conflict; defense of a concentrated population is easier than defense of a dispersed population. However, as the tactics of warfare have shifted over the centuries, the relationship between violent conflict and cities has become more complex. Their dense populations also tempt belligerents and maximize the impact to cost ratio of a given violent action. Finally, the authors point out that warfare destroys transportation infrastructure which can interfere both with commerce and with rebuilding during a conflict. 

That armed conflict near and within cities should impact national economic performance is not surprising given the existing body of literature that suggests a close relationship between urban agglomeration and national economies. \cite[p. 137]{quigley:1998} concludes that ``large cities have been and will continue to be an important source of economic growth.''  This echoes the work of Jane Jacobs who argued that cities are the primary motivators of state economies and should be given primacy over the nation state in economic analysis \citep{jacobs:1969,jacobs:1984}. In the past three decades, economists have followed Jacobs' lead and investigated city-level economic drivers \citep{lucas:1988, ciccone:hall:1996, begg:1999}. One oft-cited mechanism by which cities can provide economic advantages to industry is knowledge spillover. Industries that concentrate in cities benefit from one another through knowledge transfer \citep{jaffe:etal:1993, glaeser:1994, firestone:2010}. Empirical estimates of agglomeration effects indicate that a doubling of employment density correspond to a 5\% increase in labor productivity \citep{ciccone:hall:1996,ciccone:2002}. 

% Add some references here about cities accounting for larger portions of economies. Would be nice if we found this to be particularly true for developing countries as well. 

We extend this line of research by conceptualizing civil wars not as homogeneous national phenomenon, but as a diverse class of violent conflict with properties that distinguish the effects of one conflict from another. Since urban economies are responsible for a disproportionate share of national economic performance, civil conflicts in or near these engines of commerce should likewise exert a disproportionate influence on state performance. 

This hypothesis does not seem unknown to armed actors. The guerilla group Fuerzas Armadas Revolucionarios de Colombia (FARC) appears to have internalized these mechanisms. In 1998 and 1999, the organization moved its violent operations from mostly rural areas of Colombia into major cities and near to the capital (Petras and Brescia 2000). This coincided with economic strain caused by the implementation of an IMF/World Bank structural readjustment program. However, the timing was likely not coincidental. FARC advocates a number of political and economic reforms and chooses targets strategically related to these objectives.

Forbes magazine, reporting on peace talks between FARC guerillas and the Colombian government in 2012, wrote: 
\begin{quote}FARC's strategy and [beliefs have] always been to make economic pressure on both, multinational companies and the Colombian government. This has been done by attacking oil and natural gas infrastructure affecting companies such as Pacific Rubiales Energy, Oxy and Ecopetrol. For non-fuel related international companies with subsidiaries in Colombia, such as Goodyear, Nestle, Microsoft, Toyota, among others, FARC’s modus operandi was mainly racketeering, kidnappings and extortion. (Flannery 2012)\end{quote}
By targeting economic centers and resource infrastructure, FARC can strain Colombia's economy, frighten investors, and bolster support from poor and rural workers sensitive to wealth disparity in the country. \cite{rabasa:chalk:2001} identify a three-pronged strategy pursued by FARC in the 1990s: to consolidate power in coca-growing regions, to conduct military operations in economically valuable areas, and to isolate major cities from the rest of the country by limiting communication and travel between them.   

While FARC seems to exploit its ability to target areas of economic importance including cities, other insurgencies tend to be more peripheral. India, for instance, has faced challenges by armed groups in its north-east for half a century. However, this area of India is remote, primarily agrarian, and relatively less populous than other parts of India. Indeed, for much of this period, India has experienced relatively robust economic growth. Mexico, now nearly a decade into a violent and complicated conflict between several organized criminal enterprises and the federal government, has maintained healthy economic performance. For much of this time, the cartel violence generally occurred in rural areas along drug trafficking routes and not within major cities. \cite{beittel:2011} writes in 2011 that ``drug trafficking-related killings remain concentrated in a relatively few cities,'' but notes that the violence was beginning to expand in geographic range by 2010.

We turn now to an empirical test of our proposition. Using a dataset that includes 510 country-years of civil war from 71 different countries, we seek to determine how the geography of conflict impacts state economic outcomes.

%	Columbia +/- (BR), Pakistan - (SM), Philippines (BR), Israel (SM), Chad (BR), Democratic Republic of Congo (SM)
%	(  + = Case where conflict was close to center and economy declined; - = Case where conflict far from center and no effect on econ )

%End with a chart to show the distribution of econ performance among conflict-cntry-years to illustrate that there is noticeable variation in economic performance to explain. ending with this might help us lead into the empirics section and 
%
%	seems like the density plot you've already made in the models.R file highlights that the gdp growth var is kind of normal centered at zero which is good for us	