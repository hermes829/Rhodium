\section{Theory}
\label{theory}

% thesis: conflicts only significantly dampens economic performance if they are a threat to major population centers. conflicts isolated in sparsely populated territories of the country have little to no effect on the whole. point to the imai weinstein paper as an example of this. 

We suspect that substate factors will determine the economic impacts of civil conflict.  While lootable resources are often pointed to as sources of funding for rebel groups, these same resources are often critical to state economic performance.  The loss of safe access to these resources due to internal conflict should adversly affect a state's economy.  Companies that rely on the extraction of resources may find themselves unable to access those resources if armed conflict is proximately located to the resource sites.  Therefore, we expect state economic performance to be directly related to the distance from conflict zones to valuable resources.

Other resources are also valuable to a state's ability to conduct business.  In particular, citizens must be able and willing to participate in commerce.  When major population centers are threatened by violence, residents will be less likely to engage in economically productive activities.  Violence near major population centers not only threatens residents directly, but impedes business by threatening trade between the population center and other cities or rural areas.  For this same reason, violence near important transporation hubs such as airports and sea ports should threaten business.

Finally, the perception of violence near major cities and resource centers should negatively impact foreign investment.  Investors will react negatively to news that violence is occuring near major cities and resources.  Investment is generally contingent on the expectation of a stable labor base and, sometimes, reliable access to resources.  Conflicts that appear to threaten these, whether they do or not, should correspond to a decrease in foreign investment.\footnote{While we intend to test each of these hypotheses carefully, our project in its current form addresses only the second hypothesis - the effect of conflict distance from city centers on aggregate state economic performance.  Future iterations of this paper will also include tests of the remaining hypotheses using foreign investment, economic growth, and domestic investment as dependent variables.}

While these hypotheses may resemble those of \cite{imai:weinstein:2000}, ours differs in the hypothesized mechanism through which conflict affects economic performance.  While we do not disagree that the spread of a conflict could impact state economic prospects, we argue that conflict area is not necessary for adverse economic performance.  Conflict area is only one possible proxy for overall destructiveness.  However, conflicts with smaller spatial areas can be similarly disruptive if they are centered near (and impede access to) those resources outlined above.  In fact, we feel measures of proximity rather than spread are more appropriate to test the hypothesis that conflict obstucts vital economic activity.

These hypotheses do not seem unknown to armed actors.  The guerilla group Fuerzas Armadas Revolucionarios de Colombia (FARC) appears to have internalized these mechanisms.  In 1998 and 1999, the organization moved its violent operations from mostly rural areas of Colombia into major cities and near to the capital (Petras and Brescia 2000).  This coincided with economic strain caused by the implementation of an IMF/World Bank structural readjustment program.  However, the timing was likely not coincidental.  FARC advocates a number of political and economic reforms and chooses targets strategically related to these objectives.

Forbes magazine, reporting on peace talks between FARC guerillas and the Colombian government in 2012, wrote: 
\begin{quote}FARC's strategy and [beliefs have] always been to make economic pressure on both, multinational companies and the Colombian government. This has been done by attacking oil and natural gas infrastructure affecting companies such as Pacific Rubiales Energy, Oxy and Ecopetrol. For non-fuel related international companies with subsidiaries in Colombia, such as Goodyear, Nestle, Microsoft, Toyota, among others, FARC’s modus operandi was mainly racketeering, kidnappings and extortion. (Flannery 2012)\end{quote}
By targeting economic centers and resource infrastructure, FARC can strain Colombia's economy, frighten investors, and bolster support from poor and rural workers sensitive to wealth disparity in the country.  \cite{rabasa:chalk:2001} identify a three-pronged strategy pursued by FARC in the 1990s: to consolidate power in coca-growing regions, to conduct military operations in economically valuable areas, and to isolate major cities from the rest of the country by limiting communication and travel between them.    

While FARC seems to exploit its ability to target areas of economic importance including cities, other insurgencies tend to be more peripheral.  India, for instance, has faced challenges by armed groups in its north-east for half a century.  However, this area of India is remote, primarily agrarian, and relatively less populous than other parts of India.  Indeed, for much of this period, India has experienced relatively robust economic growth.

We turn now to an empirical test of our proposition.  Using a dataset that includes 510 instances of civil war from 71 different countries, we seek to determine how the geography of conflict impacts state economic outcomes.

%	Columbia +/- (BR), Pakistan - (SM), Philippines (BR), Israel (SM), Chad (BR), Democratic Republic of Congo (SM)
%	(  + = Case where conflict was close to center and economy declined; - = Case where conflict far from center and no effect on econ )

%End with a chart to show the distribution of econ performance among conflict-cntry-years to illustrate that there is noticeable variation in economic performance to explain. ending with this might help us lead into the empirics section and 
%
%	seems like the density plot you've already made in the models.R file highlights that the gdp growth var is kind of normal centered at zero which is good for us	