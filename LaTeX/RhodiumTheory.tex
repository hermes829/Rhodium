\section{Theory}
\label{theory}

Begin this section by showing a chart that tracks the effect of conflict duration on economic performance in the aggregate. The purpose of the chart would be to show that long running conflicts are not permanently depressing economic performance. 

thesis: conflicts only significantly dampens economic performance if they are a threat to major population centers. conflicts isolated in sparsely populated territories of the country have little to no effect on the whole. point to the imai & weinstein paper as an example of this. 

Laying out our hypothesis in a formal way

	Use the case study of Mexico/India as a way to explain our hypothesis: 

	Or instead of relying on just one case study maybe we can walk through the example for a number of countries.

	maps: would be cool if maybe we could show the progression of a certain conflict across time and space. according to our theory what we would expect to happen as a conflict gets farther away from population centers is for econ performnace to improve. on the other hand, a conflict that moves closer and closer to population centers should lead to declined in econ performance.

End with a chart to show the distribution of econ performance among conflict-cntry-years to illustrate that there is noticeable variation in economic performance to explain. ending with this might help us lead into the empirics section and 

	seems like the density plot you've already made in the models.R file highlights that the gdp growth var is kind of normal centered at zero which is good for us	