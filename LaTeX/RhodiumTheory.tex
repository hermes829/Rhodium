\section{Theory}
\label{theory}

% thesis: conflicts only significantly dampens economic performance if they are a threat to major population centers. conflicts isolated in sparsely populated territories of the country have little to no effect on the whole. point to the imai weinstein paper as an example of this. 

Laying out our hypothesis in a formal way

The guerilla group Fuerzas Armadas Revolucionarios de Colombia (FARC) appears to have to internalized these lessons.  In 1998 and 1999, the organization moved its violent operations from mostly rural areas of Colombia into major cities and near to the capital (Petras and Brescia 2000).  This coincided with economic strain caused by the implementation of an IMF/World Bank structural readjustment program.  However, the timing was likely not coincidental.  FARC advocates a number of political and economic reforms and chooses targets strategically related to these objectives.

Forbes magazine, reporting on peace talks between FARC guerillas and the Colombian government in 2012, wrote: 
\begin{quote}FARC's strategy and [beliefs have] always been to make economic pressure on both, multinational companies and the Colombian government. This has been done by attacking oil and natural gas infrastructure affecting companies such as Pacific Rubiales Energy, Oxy and Ecopetrol. For non-fuel related international companies with subsidiaries in Colombia, such as Goodyear, Nestle, Microsoft, Toyota, among others, FARC’s modus operandi was mainly racketeering, kidnappings and extortion. (Flannery 2012)\end{quote}
By targeting economic centers and resource infrastructure, FARC can strain Colombia's economy, frighten investors, and bolster support from poor and rural workers sensitive to wealth disparity in the country.  Rabasa and Chalk (2001) identify a three-pronged strategy pursued by FARC in the 1990s: to consolidate power in coca-growing regions, to conduct military operations in economically valuable areas, and to isolate major cities from the rest of the country by limiting communication and travel between them.    

%	Columbia +/- (BR), Pakistan - (SM), Philippines (BR), Israel (SM), Chad (BR), Democratic Republic of Congo (SM)
%	(  + = Case where conflict was close to center and economy declined; - = Case where conflict far from center and no effect on econ )

End with a chart to show the distribution of econ performance among conflict-cntry-years to illustrate that there is noticeable variation in economic performance to explain. ending with this might help us lead into the empirics section and 

	seems like the density plot you've already made in the models.R file highlights that the gdp growth var is kind of normal centered at zero which is good for us	