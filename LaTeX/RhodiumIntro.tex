\section{Introduction}
\label{intro}

Understanding the effects of civil conflict has become an integral part of the study of economic development \citep{serneels:verpoorten:2013}. From a theoretical perspective there is little consensus about the impact that conflict has on economic performance. While neoclassical models predict that recovery from conflict should be relatively quick, others such as \citet{barro:martin:2004} and \citet{sachs:2006} argue that recovery will be extremely slow or even trapped in a low level equilibrium. Empirical analyses have borne out no clear answers either as results on the economic ramifications of civil war vary depending on the cases and time periods included \citep{kang:meernik:2005}. 

There are also puzzling cases of countries enmeshed in domestic armed conflicts that have seemingly little impact on economic growth. Mexico, for example, has been engaged in a drug war since December 2006 and the number of deaths stemming from this civil conflict have only increased since that time. Yet, even as fatalities due to the drug war reached new peaks, Vikram Pandit of Citigroup cited Mexico in 2012 as being extremely well poised for growth. Others have argued that Mexico is actually beginning to catch up to its regional rival, Brazil \citep{vardi:2012}. This contrasts greatly with the economic fortunes of other countries in the midst of civil war where economies have plumetted. \cite{collier:elliott:etal:2003}, for instance, describe civil war as ``development in reverse.''  ``The overall economic and political legacy from civil war,'' they write, ``is thus sufficiently adverse that rapid recovery is unlikely.''  Nevertheless, some countries appear to weather domestic conflict better than others.  



% Could we say something instead like, We argue that to explain variation in economic performance between countries in the midst of civil war, it is necessary to specify the theoretical mechanism through which conflict affects growth. And the key mechanism is  that conflict harms growth by affecting cities. 

To explain this variation in economic performance between countries in the midst of civil war, it is necessary to understand how conflict affects growth. In this paper we argue that to explain variation in economic performance between countries in the midst of civil war, it is necessary to account for the spatial location of the conflict relative to major population and economic centers. Much of the extant literature has focused on examining the effect of conflict on economic growth at the national level. The implicit assumption of these approaches is that all civil wars are alike, whether they be occurring right outside the capital or on the fringes of the country. We argue that the location of conflicts to major cities and capitals will determine the effect of civil conflict on economic performance, and that conflicts only significantly dampen economic performance if they are proximate to major population and economic centers. 

The rest of the paper proceeds as follows.  We begin with a review of the literature on the relationship between economic growth and civil war. In section \ref{theory}, we explain the mechanisms of our hypothesis through the case of Colombia. In section \ref{empirics}, we describe how we construct our measures capturing distance of conflicts between major economic and population centers and lay out our estimation approach. Finally, we discuss the findings of our analysis and end with next steps. 