\section{Introduction}
\label{intro}

Puzzle: Why do some conflicts have immediate impacts on a country's economy while others that fester for decades have little to none. 	

	Negative cases (cases where there was civil war but no major declines in economic growth)

		Mexico case: high levels of violence since 2008 but little effect on economic growth. Mexico facing serious internal political stability (civil violence, civil war...) yet this has little effect on economic growth as of yet. When should we expect to see the adverse effects of civil war manifest themselves in terms of lower economic growth?

	Cases apart from Mexico:

			Pakistan - NWFP
			India
			Israel
			Russia - Chechnya
			Columbia - FAARC

	Positive cases 

			Syria
			Sudan
			Congo
			Maybe European Countries

	To tell the stories from these cases we should have a descriptive paragraph talking about a couple of cases. Then we should construct a chart that shows just two countries one from the negative case bucket and one from the positive case bucket where we have GDP per capita or GDP per capita growth on the Y-Axis and time on the X-axis. We can have separate charts for each country so that we can take into acount varying times in conflict periods. Then have another paragraph describing the plot.

Current literature has focused on examining the effect of conflict on economic growth at the national level. Additionally, much of the literature has focused on the effect of declines in economic growth on the incidence of civil war. We will argue that the analyses presented to date have been conducted at the wrong level. Analyzing the effect of civil conflict on economic growth needs to be done at a subnational level. 

Spatial Dist Hypothesis: Location of conflicts to major cities and capitals determine effect of civil conflict on economic performance. thesis: conflicts only significantly dampens economic performance if they are a threat to major population centers. conflicts isolated in sparsely populated territories of the country have little to no effect on the whole. 