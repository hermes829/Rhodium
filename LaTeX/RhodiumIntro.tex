\section{The Politics of Third-Party Intervention}
\label{intro}

When do third parties decide to intervene in civil conflict? How does this intervention affect the outcome? These questions are motivated largely by recent events in North Africa and Southwest Asia. In particular, the intervention of coalition forces in the conflict between Libyan rebels and the government of Muammar Qaddafi presented us with a puzzle. In late spring 2011, the rebel militias were in dire shape; it appeared as if they were just trying to hold on until they received outside help. The rebels' dire straits indicated to Western leaders that their window of opportunity to remove Col. Qaddafi was narrowing. 

On 17 March, 2011, the United Nations Security Council adopted Resolution 1973, which \href{http://www.un.org/News/Press/docs//2011/sc10200.doc.htm}{authorized} member states to undertake ``all necessary measures'' to protect Libyan civilians. Military operations involving air and naval forces began two days later. Almost exactly 7 months after international forces intervened, on October 20, Muammar Qaddafi was killed, effectively solidifying the rebel victory. 

Was third-party intervention necessary for rebel victory? By how much did it change the expected duration of the conflict? In this paper we develop an agent-based model of civil war focused on three actors and two variables. Governments and rebels differ in their \emph{identity} (a one-dimensional scale that could be understood as any measure of identity over which groups conflict), and the outside actor is closer to one or the other of them. In the Libya case, international powers sided with the rebels. All three actors also have certain levels of \emph{strength}, which is allocated to each party independently of the others' allocations. The outcome of each round of conflict is probabilistic based on the relative power of each side, and eventually results in the total victory of one side when it controls the entire territory. We measure three outcome variables: the percent of rebel-controlled territory at the end of the conflict, the duration of the conflict, and whether the third party chose to intervene. 

Specifically, we model the conflict as a ``gambler's ruin,'' where intervention can improve rebels' chances of winning each round, and therefore the whole conflict. Later in this section we discuss how our model relates to previous models of civil wars and insurgencies, as well as empirical evidence to support several of our assumptions. We present further details of our model in Section \ref{formal}. Section \ref{sims} discusses the results of our simulations. We find that intervention increases the chances of rebel success, but also increased the expected duration of the conflict. In Section \ref{libya} we return to the Libyan case to identify what leverage the model gives us in a counterfactual analysis. The paper concludes policy implications and a discussion of future steps in this research program in Section \ref{conclusion}.


