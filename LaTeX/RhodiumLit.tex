\subsection{Literature Review}
\label{lit}

varying hypotheses about the effect of war on economic development, war ruin v war renewal hypotheses

literature on sub-state determinants of conflict and outcome (pierskala hollenbach 2013)

	explain how what has been done in the spatial conflict literature and even the sub-nat/spat literature does not capture what we have done in this paper

% literature on endogeneity of conflict and economic performance (sambanis 2001, worldbank 2003)

% 	do we want to wade in to the endogeneity debate? i'm not sure we do that's why i've commented this out temporarily. seems to me our life would be made easier if we just tied ourselves to teh literature that puts econ growth as the dv nad conflict as the iv. u probably know better than me if this position is untenable.

\cite{collier:1999} identifies five avenues through with civil conflict can impede economic performance: through the \textit{destruction} of resources, through \textit{disruption} of social and economic activity, through \textit{diversion} of resources to the war effort, through \textit{dissaving}, and through \textit{portfolio substition} or divestment.  Of course, these mechanisms are related to one another; portfolio substitution may be exacerbated by the destruction of resources or the distruption of socioeconomic activity.  Overall, Collier finds that civil wars correspond to a 2.2 percent decrease in annual economic growth.  While he suspects that the impact will differ across economic sectors, reliable and disaggregated data was not available to test this hypothesis thoroughly.  However, preliminary evidence for this is found in their analysis of the National Accounts data of Uganda before, during, and after its civil war.

Instead of disaggregating economic outcomes, \cite{imai:weinstein:2000} instead disaggregate conflict itself.  They distinguish between those conflicts that cover larger or smaller geographic areas and hypothesize that larger conflicts (in terms of geographical spread) will result in worse economic performance.  Using a variety of regression techniques, they find that there is a negative correlation between the geographical spread of conflict and the decade average of economic growth for each country.  Widespread conflicts, they argue, are more likely to result in damage to infrastructure, divestment from normal state spending, and capital flight.  However, in subsequent simulations that account for estimation and fundamental uncertainty, the authors show that these results are uncertain and should be interpreted with caution.

