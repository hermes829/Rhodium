\subsection{Literature Review}
\label{lit}

varying hypotheses about the effect of war on economic development, war ruin v war renewal hypotheses

literature on sub-state determinants of conflict and outcome (pierskala hollenbach 2013)

	explain how what has been done in the spatial conflict literature and even the sub-nat/spat literature does not capture what we have done in this paper

% literature on endogeneity of conflict and economic performance (sambanis 2001, worldbank 2003)

% 	do we want to wade in to the endogeneity debate? i'm not sure we do that's why i've commented this out temporarily. seems to me our life would be made easier if we just tied ourselves to teh literature that puts econ growth as the dv nad conflict as the iv. u probably know better than me if this position is untenable.

\cite{imai:weinstein:2000} explored the differing economic impacts of civil war by distinguinshing between those conflicts cover larger or smaller geographic areas.  Using a variety of regression techniques, they find that there is a negative correlation between geographical spread of conflict and the decade average of economic growth for each country.  Widespread conflicts, they argue, are more likely to result in damage to infrastructure, divestment from normal state spending, and capital flight.  However, in subsequent simulations that account for estimation and fundamental uncertainty, the authors show that these results are uncertain and should be interpreted with caution.

