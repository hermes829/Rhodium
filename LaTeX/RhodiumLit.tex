\subsection{Literature Review}
\label{lit}

varying hypotheses about the effect of war on economic development, war ruin v war renewal hypotheses

literature on sub-state determinants of conflict and outcome (pierskala hollenbach 2013)

	explain how what has been done in the spatial conflict literature and even the sub-nat/spat literature does not capture what we have done in this paper

% literature on endogeneity of conflict and economic performance (sambanis 2001, worldbank 2003)

% 	do we want to wade in to the endogeneity debate? i'm not sure we do that's why i've commented this out temporarily. seems to me our life would be made easier if we just tied ourselves to teh literature that puts econ growth as the dv nad conflict as the iv. u probably know better than me if this position is untenable.

\subsubsection{Civil War $\rightarrow$ Economic Performance}

\cite{collier:1999} identifies five avenues through with civil conflict can impede economic performance: through the \textit{destruction} of resources, through \textit{disruption} of social and economic activity, through \textit{diversion} of resources to the war effort, through \textit{dissaving}, and through \textit{portfolio substition} or divestment.  Of course, these mechanisms are related to one another; portfolio substitution may be exacerbated by the destruction of resources or the distruption of socioeconomic activity.  Overall, Collier finds that civil wars correspond to a 2.2 percent decrease in annual economic growth.  While he suspects that the impact will differ across economic sectors, reliable and disaggregated data was not available to test this hypothesis thoroughly.  However, preliminary evidence for this is found in their analysis of the National Accounts data of Uganda before, during, and after its civil war.

Instead of disaggregating economic outcomes, \cite{imai:weinstein:2000} instead disaggregate conflict itself.  They distinguish between those conflicts that cover larger or smaller geographic areas and hypothesize that larger conflicts (in terms of geographical spread) will result in worse economic performance.  Using a variety of regression techniques, they find that there is a negative correlation between the geographical spread of conflict and the decade average of economic growth for each country.  Widespread conflicts, they argue, are more likely to result in damage to infrastructure, divestment from normal state spending, and capital flight.  However, in subsequent simulations that account for estimation and fundamental uncertainty, the authors show that these results are uncertain and should be interpreted with caution.

That civil wars negatively impact economic performance, while in line with the ``war ruin'' hypothesis, runs counter to the ``war renewal'' hypothesis.  Some scholars have argued that wars, international wars in particular, can spur economic development\footnote{For a review of this discussion, see \cite{rasler:thompson:1985}}.  The prevailing wisdom with regard to civil war, however, is that outcomes of this nature are the exception rather than the rule.  In a test of economic and social determinants of post-conflict recovery in the context of civil war, \cite{kang:meernick:2005} find that these conflicts can lead, under different conditions, to either rapid or stagnant economic recovery.  They conclude that the long-term economic impacts of civil war are largely dependent on post-war governance and foreign assistance.  They also find, perhaps unsuprisingly, that aggregate estimates of conflict destructiveness are negatively correlated with long-term growth.

Not only do several studies link civil war to domestic economic performance, there is also evidence that civil wars have regional economic consequences.  \cite{murdoch:sandler:2002} find some evidence that states neighboring civil war states are more likely to experience poor short-term economic performance.  They attribute this effect to the disruption of trade and uncertainty about the potential for conflict to spread across the border.

\subsubsection{Economic Performance $\rightarrow$ Civil War}

Much work has been done on the causal effects of economic performance on civil war.  Indeed, there is likely an endogenous relationship between economic performance and civil war; each exacerbates the other.  While our work here sidesteps this argument by focusing exclusively on observations of civil war, we will briefly review the relevant literature here.  In a report for the World Bank by \cite{collier:etal:2003}, the authors describe what they term the \textit{conflict trap}.  States that find themselves in the \textit{conflict trap} are those that have experienced civil war with, are subsequently affected by its economic and social consequences, and are therefore more likely to experience further civil conflict.  During civil wars, resources are diverted from productive economic activity to destructive activity.  These diverted resources act to stall progress during the conflict and are often used to destroy the infrastructure necessary for growth afterwards.  These changes to economic performance, as well as structural changes to the economy itself, make the resurgence of war more likely.  

In accord with this theory, \cite{fearon:laitin:2003} aruge that poor economic growth is the primary condition conducive for civil war.  More specifically, they believe that strong economic growth proxies for robust governance and that states with low GDP growth likely have infrastructures that are unable to implement counterinsurgent policies.  In an effort to parse out the causal effect of economic growth shocks on civil war, \cite{miguel:etal:2004} instrument income growth with rainfall.  They find that rainfall is strongly correlated with income in sub-Saharan Africa, a region also prone to civil conflict in recent decades.   Using a two-stage estimation approach, they find that rainfall, their exogenous instrument for income, is positively correlated with the likelihood of civil war.

\subsubsection{Disaggregating Civil Wars}

Recently, scholars have begun to spatially disaggregate civil conflicts.  New data allows researchers to focus on how the geography of internal conflict varies.  




