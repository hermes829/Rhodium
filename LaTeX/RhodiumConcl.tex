\section{Conclusion}
\label{conclusion}

Our project seeks to explain observed discrepencies in the economic impact of civil conflict.  Violent intrastate conflicts have occured in one third of countries in just the past several decades.  While civil wars seem to primarily afflict developing states, their actual economic impacts are still subject to debate.  Some states weather civil conflict for years or even decades while simultaneously prospering economically.  Other states see their economies stall and suffer substantial declines in growth.  Despite this heterogeneity observed among civil war states, the individual characteristics of civil conflict that lead to disparate outcomes have only recently been subject to academic scrutiny.  In this first step of a larger project to distinguish the geography of civil conflict, we hope to have contributed to our understanding of why some civil conflicts impact economic performance more severely than others.  We have used a novel approach in the study of civil conflict to distinguish between spatially dissimilar events and shown that this new measure of interest, the minimum distance between conflict epicenters and major population centers, is a substantive determinant of economic growth.  While the proximity of conflict to major cities helps us to better understand how domestic armed conflict produces disparate economic outcomes across countries, one can imagine a number of other hypotheses that derive from this research.

Population centers are critical to economic performance, but still tell only a part of the story.  States rely on various natural resources as well as critical infrastructure to prosper.  Trade depends on safe and reliable access to ports, airports, and railways.  Businesses rely on safe and reliable access to natural resources.  And foreign direct investment relies on the investor's perception of state stability.  While there is a large literature on natural resources and conflict at the state level, little work has disaggregated this data and explored the micro-relationship between conflict and resources.  Future iterations of this paper will explore each of these hypotheses using geospatial data on the economic assets of interest.

Looking beyond aggregate measures of state economic performance, we hope to further explore how the geography of conflict impacts the surrounding region.  A growing body of work suggests that civil conflict is, in and of itself, contagious.  However, the precise mechanisms that determine contagion risk are still unclear.  We propose that looking at the location of conflict within a state will shed light on how conflicts and their economic effects spread across borders.  The same logic that applies to the hypotheses presented in this paper applies to neighboring states as well.  Resources and population centers that are near the border between a peaceful state and a civil war state may be impacted in the same way that resources and cities within the conflict state are.  This may help to explain the disparate regional economic effects of civil war observed by \cite{murdoch:sandler:2002}.