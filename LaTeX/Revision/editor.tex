\section{Editor}


In revising your paper you should consider the full range of questions raised by each of the reviewers as they offer thoughtful comments and suggestions for improving the paper. I want to highlight some of those points to emphasize the need for your careful attention to them as you work on revisions.

Regarding R1, there are three issues to highlight. 
\begin{enumerate}
\item  First, this reviewer would like to see your main model be run on a fixed effects setup with standard errors clustered on country. \\

\textcolor{blue}{\emph{
	There are a number of reasons why we chose a random effects framework, however, we agree with the reviewers that these reasons were not clearly explicated in the text. The following discussion provides three reasons why we employ a random effects framework. We have added the reasoning described below to the relevant section in the actual paper as well.
	For the ``Capital City'' and ``Any Major City'' models we ran a Hausman test to see if fixed or random effects were appropriate for our dataset, and in both cases we do not reject the null hypothesis of the Hausman test at a 90 or 95\% confidence interval indicating by at least that standard that both fixed and random effect approaches are consistent, with the latter obviously being more efficient.
	However, as Clark \& Linzer (2015) note the Hausman test should not be the sole determination for choosing between fixed or random effects. They perform a series of Monte Carlo simulations to determine the conditions under which a fixed or random effects model is appropriate and provide a rough typology. Specifically to take into consideration the size of the dataset (both number of units, in this case countries, and number of observations per unit, conflict instances) and the level of correlation between the regressor and unit effects. In our case, we have over 70 countries but for over half of those we only have five conflict instances or less. Given such a data structure Clark \& Linzer (2015) recommend to examine the level of correlation between the regressor and unit effects to determine the appropriate modeling framework. For both our distance models, the level of correlation between the regressor and unit effects is less than 0.20, which accords with a random effects recommendation under the framework described by Clark \& Linzer (2015). 
	Another reason why we choose a random effects framework is because of our concerns with the unchanging and time invariant nature of where conflict is taking place relative to major urban centers. In Thailand, for example, the distance between conflict and urban centers in our dataset just ranges from approximately 790 to 810 kilometers, which basically indicates that conflicts are simply isolated to a specific part of the country. This same patterns holds for many other countries in our sample such as Mozambique (range: $\approx$ 705 - 860 km), Bangladesh (range: $\approx$ 198 - 237 km), Cambodia (range: $\approx$ 131 - 197 km), etc.. These ranges become even further compressed when we log them for use in our regression analysis. If we employed a fixed effects model to test our hypothesis we would in essence be removing many of these types of countries from our sample, or as Beck \& Katz (2001) would put it ``throwing out the baby with the bathwater''. 
}}

\item Second, this reviewer would like to see country year observations without conflict included in the analyses instead of being excluded. I should note that R2 raises these two concerns as well which suggests that it will be critical to address these issues in your revisions. \\

\textcolor{blue}{\emph{
	Including an interaction in this way is problematic for estimating the effect of conflict distance on conflict. The logged, minimum distance variable ranges from approximately 0.33 to 7.31, with closer values indicating that conflict is more proximate to an urban center for that country-conflict-year. This variable is NA for cases in which no conflict occurred for a given country year, thus before we interact it with the civil war variable in a full panel set up we would need to introduce some values for the NAs. One possibility is to simply invert the distance variable and then set the NAs to zero. However, the choice of introducing a zero for the NAs would be arbitrary and problematic, as our variable would no longer truly be continuous. Instead it would be discontinuous in that we would have a large lump of observations at zero and then no observations until 1/7.31. Beyond the problems associated with transforming the variable in this way we would also run into issues of perfect collinearity with the inverted conflict distance measure and the interaction variable. The collinearity would result because the binary conflict variable is zero for every case that the inverted conflict distance variable is zero and one otherwise, meaning that multiplying the two will simply result in the inverted conflict distance variable again. 
	We have spent a fair bit of time trying to think of alternative approaches to modeling this in a full panel context using mixture or hierarchical approaches, but could find no mentions to develop a model for this type of data. 
	Despite this we strongly feel that our results are illustrative of an important and meaningful finding for the conflict literature. There has been very little discussion of the role that the spatial distribution of conflict plays in shaping macroeconomic outcomes, and the findings that we present here are the first to begin to disentangle this relationship.
}}

\item Third, this reviewer would like to see you estimate a baseline model for the impact of conflict on growth in which GDP growth is regressed on a dummy for armed conflict in a country. After that, you can then move onto your central argument and compare your findings to this baseline set of findings. \\

\textcolor{blue}{\emph{
	We have run a model on our full panel dataset to calculate a benchmark on the effect of any conflict on GDP growth. The results of this analysis have been added to the paper. To make the baseline model as comparable as possible to the model that we use to test our distance hypothesis, we adopt as similar a specification as possible. Specifically, the dependent variable is again GDP growth. Then our independent variables are, each lagged by one year: 
	\begin{itemize}
		\item Civil war, this is just a binary to indicate whether a civil war took place in that year
		\item Upperincome, this is a binary to indicate whether the country is classified as upper or lower income according to the Worldbank
		\item Logged, inflation
		\item Polity score
		\item Resources as a percent of GDP
		\item Average GDP growth across the world for that year
	\end{itemize}
	We do not include the conflict-specific measures (Distance, Intensity, Duration, and Area) from our distance model since they would just have a value of NA during non-conflict years. The key finding from this analysis is that a civil war is related to lower levels of GDP growth, however, this effect is marginal and only significant at a 90\% confidence interval. To determine the substantive significance of a civil war we employ a simulation based approach, similar to what we did to assess the effect of conflict distance from urban centers on growth, and we find that there is a marginal difference but the effect is highly uncertain. We use this finding as a lead in to our main model describing the effect of conflict distance from urban centers on growth. 
}}

\end{enumerate}

Regarding R2, there is one additional issue to highlight (points one and two overlap with R1).

\begin{enumerate}
\item The third point raised is to run a robustness check on the more disaggregated ACLED data set to see if your findings hold up. \\

\textcolor{blue}{\emph{
	We retested our hypothesis using the ACLED dataset and we find that our distance measure of conflict to major urban centers is significantly negative. We have included these results in the Appendix under the subsection ACLED Analysis, we have also added in a footnote in the paper indicating that our results remain robust when estimated on this alternative dataset.
}}

\end{enumerate}