\documentclass[12pt,onesided,fullpage]{amsart}

%%% PAGE DIMENSIONS
\usepackage{multirow}
\usepackage[top=1in, bottom=1in, left=1in, right=1in]{geometry} 
%\pdfpagewidth=8.5in % for pdflatex
%\pdfpageheight=11in % for pdflatex


\usepackage{graphicx} % support the \includegraphics command and options

%%% PACKAGES
\usepackage{booktabs} % for much better looking tables
\usepackage{array} % for better arrays (eg matrices) in maths
\usepackage{paralist} % very flexible & customisable lists (eg. enumerate/itemize, etc.)
\usepackage{verbatim} % adds environment for commenting out blocks of text & for better verbatim

%%% HEADERS & FOOTERS
\usepackage{fancyhdr} % This should be set AFTER setting up the page geometry
\pagestyle{fancy} % options: empty , plain , fancy
\renewcommand{\headrulewidth}{0pt} % customise the layout...

\usepackage{amsfonts}
\usepackage{amsmath}
\usepackage{amssymb}
\usepackage{array,amsmath,graphicx,psfrag,amssymb,subfigure,tabularx,booktabs}
\usepackage{mparhack}
\usepackage{setspace}
\usepackage{natbib}
\usepackage{multicol}
\usepackage{color}
\usepackage{dcolumn}
\usepackage{hyperref}
\usepackage{stmaryrd}
\setcitestyle{authordate,round,semicolon,aysep={,},yysep={,}}
\bibpunct[:]{(}{)}{;}{a}{,}{,}

\doublespacing

\begin{document}
\singlespacing

\title[JCR-14-0358​]{JCR-14-0358​ entitled ``Enemy at the Gates: Variation in Economic Growth from Civil Conflict''}

\date{\today~~Version 0.01}
\maketitle

$\hspace{-5mm}$Dear Professor Huth, \\ [1ex]

We first would like to thank you for the opportunity to revise and resubmit our manuscript. We believe the manuscript has greatly benefitted from the Reviewers' helpful and thoughtful comments. We have thoroughly revised the manuscript, taking seriously each individual point raised by the Reviewers. The revision memo is organized by first responding to your comments and then addressing the reviewers' points. Our comments and responses are shown in \textcolor{blue}{\emph{BLUE}} below each point.

We hope you agree that the manuscript has greatly improved through this helpful process and we are looking forward to your response.\\ [1ex]

Sincerely, \\ [1ex]

The Authors.

\section{Editor}


In revising your paper you should consider the full range of questions raised by each of the reviewers as they offer thoughtful comments and suggestions for improving the paper. I want to highlight some of those points to emphasize the need for your careful attention to them as you work on revisions.

Regarding R1, there are three issues to highlight. 
\begin{enumerate}
\item  First, this reviewer would like to see your main model be run on a fixed effects setup with standard errors clustered on country. \\

\textcolor{blue}{\emph{stuff}}

\item Second, this reviewer would like to see country year observations without conflict included in the analyses instead of being excluded. I should note that R2 raises these two concerns as well which suggests that it will be critical to address these issues in your revisions. \\

\textcolor{blue}{\emph{stuff}}

\item Third, this reviewer would like to see you estimate a baseline model for the impact of conflict on growth in which GDP growth is regressed on a dummy for armed conflict in a country. After that, you can then move onto your central argument and compare your findings to this baseline set of findings. \\

\textcolor{blue}{\emph{stuff}}

\end{enumerate}

Regarding R2, there is one additional issue to highlight (points one and two overlap with R1).

\begin{enumerate}
\item The third point raised is to run a robustness check on the more disaggregated ACLED data set to see if your findings hold up. \\

\textcolor{blue}{\emph{stuff}}

\end{enumerate}



\section{Reviewer: 1}

This paper advances the hypothesis that conflict should exert larger negative impacts on GDP if it located closer to urban centers. This idea is tested using subnational data on conflict location from the PRIO Conflict Site Dataset covering the 1989-2008 period. The authors present evidence for their hypothesis using a random effects model. Robustness checks are conducted verifying the results are not driven by a particular sub-sample.

I very much like the conceptual hypothesis advanced by this paper, and the idea of using the conflict Site Dataset to test it. However, I am concerned about the empirical specification used for the analysis. Below I outline alternative approaches. If these can be implemented, I believe the paper can make a valuable contribution to the conflict literature.


\subsection{Major comments}

\begin{enumerate}
\item The random effects model is only appropriate compared to the fixed effects model if there is a compelling reason to believe that time invariant country characteristics are uncorrelated with spatial conflict location and income. The authors don’t present a Hausman test to try and make the case for random effects. Moreover, this doesn’t seem plausible because geographic factors such as roughness of terrain and climactic conditions in remote areas far from urban centers would determine the cost of launching an insurgency in urban vs. rural areas and also affect GDP. In addition, institutional features such as whether rule of law or state presence extends to rural areas would similarly act as potential omitted variables if country fixed effects are not included. A research question along these lines really should exploit variation within a country and show that as urban conflict location chances, GDP growth changes, instead of relying on comparisons across countries. As such, I would want to see the results hold with the fixed effects model as the primary model, including country and year fixed effects, and with standard errors clustered on country to control serial correlation over time.

\begin{itemize}
\item \textcolor{blue}{\emph{stuff}}
\end{itemize}

\item Currently country year observations without conflict are excluded from current specifications. They should not be because if low GDP correlates with periods of no conflict, as might arise if sufficient economic activity is required to finance internal conflicts, then this omission would lead to an upward bias on the estimates.

\begin{itemize}
\item \textcolor{blue}{\emph{stuff}}
\end{itemize}

\item It will be fairly important to benchmark the results based on distance between conflict location and urban center against the overall effect of conflict. In other words, we’d like to know if having any conflict lowers GDP, and if this effect is even stronger if its located closer to urban areas. To capture this, the authors should start with a simple specification that regresses GDP growth on a dummy for if any conflict occurred. Then they should introduce an interaction between this dummy variable and the conflict distance variable. The coefficient on the dummy in this second specification would then capture effects of 0 distance, or if conflicts occurred in cities.

\begin{itemize}
\item \textcolor{blue}{\emph{stuff}}
\end{itemize}

\item It wasn’t clear to me how 0 distance was dealt with the current specifications since logs were taken but this would eliminate the 0s. It would be very important to include these if in fact conflict in cities are represented by 0 distance in the data.

\begin{itemize}
\item \textcolor{blue}{\emph{stuff}}
\end{itemize}

\item Instead of treating conflict intensity as a control, it would be useful to re-do the specifications in point 3 above but just restricting to the incidents coded as wars and then restricting to the lower intensity effects. Then the authors can test to see if coefficients on the former are in fact larger.

\begin{itemize}
\item \textcolor{blue}{\emph{stuff}}
\end{itemize}

\item More information should be provided on how conflict events make it into the Conflict Site Dataset. For example, Figure 1 suggests too sparse picture of violent events in Colombia, given that nearly 1/3rd of the Colombian territory was war affected over this period.

\begin{itemize}
\item \textcolor{blue}{\emph{stuff}}
\end{itemize}

\item I like the use of the descriptive cases but more should be done with these if they will be included. In particular, the economic dynamic of these countries should be mentioned to clarify how the effect of conflict on GDP maps to the hypotheses advanced in the paper. I also wondered about the focus on conflict in the NE in India. Naxalite conflict has also spread rapidly and is also located in rural areas (which is consistent with the authors’ point), so this would be natural to draw upon as well.

\begin{itemize}
\item \textcolor{blue}{\emph{stuff}}
\end{itemize}

\end{enumerate}

\section{Reviewer: 2}
The authors claim that the distance between cities (or capital) and the location of violent events during a civil conflict is a first order argument to explain the effect of conflict on economic growth. I’m fully convinced of the interest of the question to understand the legacies of civil conflict. Empirical evidence on the effect of civil conflict on growth is notably crucial for the post- recovery policies. While I find the the question very interesting, I think there is a number of issues that remain to be addressed.

\subsection{Major Comments}

\begin{enumerate}

\item  I’m convinced by the interpretation of the results but I can imagine also an alternative story. The state capacity is negatively correlated to the distance (to the capital or main cities - see Buhaug, 2010). That means that the fighting cost for a rebel group is decreasing with the distance to the capital. In other words, it is only the strongest (richest) rebel groups that are able to be close enough to the capital. The effect detected in the paper is perhaps only the effect of size groups. The biggest groups are the more violent, the more disruptive and consequently that have an higher effect on economic growth. I’m not sure how it will be possible to arbitrate between this alternative story and the story of the authors. At the end, a discussion around this alternative explanation would be profitable to the paper.

\begin{itemize}
\item \textcolor{blue}{\emph{stuff}}
\end{itemize}

\item he empirical strategy used is a major issue and the authors have to deeply improved this part. The authors mainly use a cross-country comparison using random effects and they justify the use of random effects because their purpose is to explain variation between unit. But the cross-country comparison doesn’t involve the use of random effects. The authors missed also to control for time-specific shocks that are common to all countries by including year dummies. The year dummies will absorb yearly worldwide changes such as economic shocks, global climate shocks or natural resource price shocks. I don’t believe also the explanation to avoid the use of country fixed effects. I think the authors should follow the following road map:
\begin{itemize}
\item Simple correlation between GDP growth and the distance of conflict.
\item Inclusion of the control variables.
\item Inclusion of year fixed effects.
\item Inclusion of both year and country fixed effects.
\end{itemize}

\begin{itemize}
\item \textcolor{blue}{\emph{stuff}}
\end{itemize}

\item Sample. I’m not convinced by the sample choice. I would like to see results with a full sample of countries from 1989 to 2008, including peace countries. I expect the author to interact their measure of distance to conflict with a dummy coded 1 for country in civil conflict and 0 otherwise

\begin{itemize}
\item \textcolor{blue}{\emph{stuff}}
\end{itemize}

\item Data on conflicts. One could imagine that the quality/quantity of reports on conflicts is negatively correlated to this distance to the capital. I would appreciate a discussion on the potential report bias and how it could influence the results. As a robustness, I expect the authors to use ACLED data that are commonly used now as data for disaggregated analysis.

\begin{itemize}
\item \textcolor{blue}{\emph{stuff}}
\end{itemize}

\item To avoid issues linked to reverse causation, I would consider the list of urban centers
at the beginning of the period.

\begin{itemize}
\item \textcolor{blue}{\emph{stuff}}
\end{itemize}

\item Definition of variables. The authors take the minimum distance to the conflict as explanatory variable. I would like to see alternative measures to ensure that the results are not sensitive to the definition of the main variable. For instance, the weighted (by the distance) sum of the number of events is a credible candidate as an alternative measure.

\begin{itemize}
\item \textcolor{blue}{\emph{stuff}}
\end{itemize}

\end{enumerate}

\subsection{Minor Comments}

\begin{enumerate}

\item Introduction. The example on Mexico looks strange. By many aspects, the drug war in Mexico is very different to conflicts in Republic Democratic of Congo or in Uganda. In other words, it is difficult to compare conflicts with genocide, massive internal migration, ethnic cleavages with a drug war where almost all citizens support the government. The recent case of Nigeria and Cameroon with Boko Haram looks to be a better fit with the story of the authors.

\begin{itemize}
\item \textcolor{blue}{\emph{stuff}}
\end{itemize}

\item The authors claim they focus on the proximities of conflict to cities and not on the area covered by the conflict. I’m wondering whether the effect of the proximities of conflict to cities would be intensify by the area of the conflict. I expect the effect of the proximities of conflict to cities to be higher if the area of conflict is biggest. A very simple interaction term between distance and area would be appropriate to uncover this mechanism.

\begin{itemize}
\item \textcolor{blue}{\emph{stuff}}
\end{itemize}

\item Results. Conflict duration and the number of conflicts a country is facing have an unexpected effect on growth. I would see a discussion to explain this results.

\begin{itemize}
\item \textcolor{blue}{\emph{stuff}}
\end{itemize}

\item Area covered by conflict. I don’t understand why the authors use a binary measure instead of the continuous variable.

\begin{itemize}
\item \textcolor{blue}{\emph{stuff}}
\end{itemize}

\item Robustness. The strategy that consists to run a six-fold cross-validation is convincing. In the same spirit, I would like to see the same exercise when one country is left out of the sample.

\begin{itemize}
\item \textcolor{blue}{\emph{stuff}}
\end{itemize}

\item Figures have to be self contained.

\begin{itemize}
\item \textcolor{blue}{\emph{stuff}}
\end{itemize}

\item The visual presentation of the results are interesting but the classical way to present results (through tables) is a requirement.

\begin{itemize}
\item \textcolor{blue}{\emph{stuff}}
\end{itemize}

\item The recent works on cities of Quoc-Anh Do seems to me as a complement to this paper.

\begin{itemize}
\item \textcolor{blue}{\emph{stuff}}
\end{itemize}

\end{enumerate}

\newpage\tiny
\end{document}\bye