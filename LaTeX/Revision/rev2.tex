\section{Reviewer: 2}
The authors claim that the distance between cities (or capital) and the location of violent events during a civil conflict is a first order argument to explain the effect of conflict on economic growth. I’m fully convinced of the interest of the question to understand the legacies of civil conflict. Empirical evidence on the effect of civil conflict on growth is notably crucial for the post- recovery policies. While I find the the question very interesting, I think there is a number of issues that remain to be addressed.

\subsection{Major Comments}

\begin{enumerate}

\item  I’m convinced by the interpretation of the results but I can imagine also an alternative story. The state capacity is negatively correlated to the distance (to the capital or main cities - see Buhaug, 2010). That means that the fighting cost for a rebel group is decreasing with the distance to the capital. In other words, it is only the strongest (richest) rebel groups that are able to be close enough to the capital. The effect detected in the paper is perhaps only the effect of size groups. The biggest groups are the more violent, the more disruptive and consequently that have an higher effect on economic growth. I’m not sure how it will be possible to arbitrate between this alternative story and the story of the authors. At the end, a discussion around this alternative explanation would be profitable to the paper.

\begin{itemize}
\item \textcolor{blue}{\emph{
	We provide a discussion of this alternative explanation following our empirical analysis.
}}
\end{itemize}

\item The empirical strategy used is a major issue and the authors have to deeply improved this part. The authors mainly use a cross-country comparison using random effects and they justify the use of random effects because their purpose is to explain variation between unit. But the cross-country comparison doesn’t involve the use of random effects. The authors missed also to control for time-specific shocks that are common to all countries by including year dummies. The year dummies will absorb yearly worldwide changes such as economic shocks, global climate shocks or natural resource price shocks. I don’t believe also the explanation to avoid the use of country fixed effects. I think the authors should follow the following road map:
\begin{itemize}
\item Simple correlation between GDP growth and the distance of conflict.
\item Inclusion of the control variables.
\item Inclusion of year fixed effects.
\item Inclusion of both year and country fixed effects.
\end{itemize}

\begin{itemize}
\item \textcolor{blue}{\emph{
	There are a number of reasons why we chose a random effects framework, however, we agree with the reviewers that these reasons were not clearly explicated in the text. The following discussion provides three reasons why we employ a random effects framework. We have added the reasoning described below to the relevant section in the actual paper as well.
	For the ``Capital City'' and ``Any Major City'' models we ran a Hausman test to see if fixed or random effects were appropriate for our dataset, and in both cases we do not reject the null hypothesis of the Hausman test at a 90 or 95\% confidence interval indicating by at least that standard that both fixed and random effect approaches are consistent, with the latter obviously being more efficient.
	However, as Clark \& Linzer (2015) note the Hausman test should not be the sole determination for choosing between fixed or random effects. They perform a series of Monte Carlo simulations to determine the conditions under which a fixed or random effects model is appropriate and provide a rough typology. Specifically to take into consideration the size of the dataset (both number of units, in this case countries, and number of observations per unit, conflict instances) and the level of correlation between the regressor and unit effects. In our case, we have over 70 countries but for over half of those we only have five conflict instances or less. Given such a data structure Clark \& Linzer (2015) recommend to examine the level of correlation between the regressor and unit effects to determine the appropriate modeling framework. For both our distance models, the level of correlation between the regressor and unit effects is less than 0.20, which accords with a random effects recommendation under the framework described by Clark \& Linzer (2015). 
	Another reason why we choose a random effects framework is because of our concerns with the unchanging and time invariant nature of where conflict is taking place relative to major urban centers. In Thailand, for example, the distance between conflict and urban centers in our dataset just ranges from approximately 790 to 810 kilometers, which basically indicates that conflicts are simply isolated to a specific part of the country. This same patterns holds for many other countries in our sample such as Mozambique (range: $\approx$ 705 - 860 km), Bangladesh (range: $\approx$ 198 - 237 km), Cambodia (range: $\approx$ 131 - 197 km), etc.. These ranges become even further compressed when we log them for use in our regression analysis. If we employed a fixed effects model to test our hypothesis we would in essence be removing many of these types of countries from our sample, or as Beck \& Katz (2001) would put it ``throwing out the baby with the bathwater''. 
	In terms of the roadmap mentioned by the reviewer, we have added an appendix item that runs the analysis in the way reviewer suggested and reports the results in a tabular format. 
}}
\end{itemize}

\item Sample. I’m not convinced by the sample choice. I would like to see results with a full sample of countries from 1989 to 2008, including peace countries. I expect the author to interact their measure of distance to conflict with a dummy coded 1 for country in civil conflict and 0 otherwise

\begin{itemize}
\item \textcolor{blue}{\emph{
	Including an interaction in this way is problematic for estimating the effect of conflict distance on conflict. The logged, minimum distance variable ranges from approximately 0.33 to 7.31, with closer values indicating that conflict is more proximate to an urban center for that country-conflict-year. This variable is NA for cases in which no conflict occurred for a given country year, thus before we interact it with the civil war variable in a full panel set up we would need to introduce some values for the NAs. One possibility is to simply invert the distance variable and then set the NAs to zero. However, the choice of introducing a zero for the NAs would be arbitrary and problematic, as our variable would no longer truly be continuous. Instead it would be discontinuous in that we would have a large lump of observations at zero and then no observations until 1/7.31. Beyond the problems associated with transforming the variable in this way we would also run into issues of perfect collinearity with the inverted conflict distance measure and the interaction variable. The collinearity would result because the binary conflict variable is zero for every case that the inverted conflict distance variable is zero and one otherwise, meaning that multiplying the two will simply result in the inverted conflict distance variable again. 
	We have spent a fair bit of time trying to think of alternative approaches to modeling this in a full panel context using mixture or hierarchical approaches, but could find no mentions to develop a model for this type of data. 
	Despite this we strongly feel that our results are illustrative of an important and meaningful finding for the conflict literature. There has been very little discussion of the role that the spatial distribution of conflict plays in shaping macroeconomic outcomes, and the findings that we present here are the first to begin to disentangle this relationship.
}}
\end{itemize}

\item Data on conflicts. One could imagine that the quality/quantity of reports on conflicts is negatively correlated to this distance to the capital. I would appreciate a discussion on the potential report bias and how it could influence the results. As a robustness, I expect the authors to use ACLED data that are commonly used now as data for disaggregated analysis.

\begin{itemize}
\item \textcolor{blue}{\emph{
	We retested our hypothesis using the ACLED dataset and we find that our distance measure of conflict to major urban centers is significantly negative. We have included these results in the Appendix under the subsection ACLED Analysis, we have also added in a footnote in the paper indicating that our results remain robust when estimated on this alternative dataset.
}}
\end{itemize}

\item To avoid issues linked to reverse causation, I would consider the list of urban centers at the beginning of the period.

\begin{itemize}
\item \textcolor{blue}{\emph{
We reran our model on the effect of conflict proximity on GDP growth using the list of urban centers at the beginning of the period. The results are shown in table \ref{tab:modOrigCity} below. Running the same analysis using capital cities is difficult as they change over time in our sample.
% Table created by stargazer v.5.1 by Marek Hlavac, Harvard University. E-mail: hlavac at fas.harvard.edu
% Date and time: Wed, Aug 19, 2015 - 14:30:29
\begin{table}[!htbp] \centering 
  \caption{Rerunning model on the effect of conflict proximity using list of urban centers at the beginning of the period.} 
  \label{tab:modOrigCity} 
\begin{tabular}{@{\extracolsep{5pt}}lc} 
\\[-1.8ex]\hline 
\hline \\[-1.8ex] 
 & \multicolumn{1}{c}{\textit{Dependent variable:}} \\ 
\cline{2-2} 
\\[-1.8ex] & $\Delta$ GDP$_{t}$ \\ 
\hline \\[-1.8ex] 
 Ln(Min. City Dist.)$_{t-1}$ & 1.144$^{***}$ \\ 
  & (0.424) \\ 
  & \\ 
 Intensity$_{t-1}$ & $-$1.197 \\ 
  & (0.973) \\ 
  & \\ 
 Duration$_{t-1}$ & 0.141$^{***}$ \\ 
  & (0.037) \\ 
  & \\ 
 Area$_{t-1}$ & $-$4.965$^{***}$ \\ 
  & (1.283) \\ 
  & \\ 
 Number of conflicts$_{t-1}$ & 1.475$^{**}$ \\ 
  & (0.624) \\ 
  & \\ 
 Upper Income & 1.038 \\ 
  & (2.742) \\ 
  & \\ 
 Ln(Inflation)$_{t-1}$ & $-$2.586$^{***}$ \\ 
  & (0.471) \\ 
  & \\ 
 Democracy$_{t-1}$ & $-$0.008 \\ 
  & (0.091) \\ 
  & \\ 
 Resource Rents/GDP$_{t-1}$ & 0.073$^{**}$ \\ 
  & (0.036) \\ 
  & \\ 
 World GDP Growth$_{t}$ & 0.549$^{**}$ \\ 
  & (0.271) \\ 
  & \\ 
 Intercept & 1.688 \\ 
  & (3.455) \\ 
  & \\ 
\hline \\[-1.8ex] 
Countries & 69 \\
Observations & 505 \\ 
\hline 
\hline \\[-1.8ex] 
\textit{Note:}  & \multicolumn{1}{r}{$^{*}$p$<$0.1; $^{**}$p$<$0.05; $^{***}$p$<$0.01} \\ 
\end{tabular} 
\end{table} 
\FloatBarrier
}}
\end{itemize}

\item Definition of variables. The authors take the minimum distance to the conflict as explanatory variable. I would like to see alternative measures to ensure that the results are not sensitive to the definition of the main variable. For instance, the weighted (by the distance) sum of the number of events is a credible candidate as an alternative measure.

\begin{itemize}
\item \textcolor{blue}{\emph{
	We have looked at alternative parameterizations of the conflict distance measure including both the minimum and mean distances of conflicts to the respective nearest cities. Our results are robust across these parameterizations. We specifically chose not to create an aggregate conflict variable weighted by distance as we lacked theoretical justification for doing so given the argument advanced in the paper. If a country experiences both a rural and an urban conflict simultaneously, we do not expect the rural conflict to mitigate the economic consequences of the urban conflict. In the same way, we do not expect distant cities to compensate for the economic consequences felt by a city near conflict.
}}
\end{itemize}

\end{enumerate}

\subsection{Minor Comments}

\begin{enumerate}

\item Introduction. The example on Mexico looks strange. By many aspects, the drug war in Mexico is very different to conflicts in Republic Democratic of Congo or in Uganda. In other words, it is difficult to compare conflicts with genocide, massive internal migration, ethnic cleavages with a drug war where almost all citizens support the government. The recent case of Nigeria and Cameroon with Boko Haram looks to be a better fit with the story of the authors.

\begin{itemize}
\item \textcolor{blue}{\emph{
	We appreciate this suggestion and have included the cases of Nigeria and Cameroon rather than profiling Mexico at the lead of this paper. 
}}
\end{itemize}

\item The authors claim they focus on the proximities of conflict to cities and not on the area covered by the conflict. I’m wondering whether the effect of the proximities of conflict to cities would be intensify by the area of the conflict. I expect the effect of the proximities of conflict to cities to be higher if the area of conflict is biggest. A very simple interaction term between distance and area would be appropriate to uncover this mechanism.

\begin{itemize}
\item \textcolor{blue}{\emph{
	We agree that this is an interesting hypothesis and would constitute a valuable extension to this research. However, we believe that this hypothesis is not a clear corollary to the arguments presented in this paper and deserves more consideration than a simple interaction term. Conflict area is not necessarily a proxy for conflict intensity and it is unclear to us that, controlling for intensity as we do, conflicts with larger areas that are nearer to cities will be of necessarily greater economic impact. In fact, one might argue that there is a dilution effect in which larger conflicts, controlling for intensity, produce less violence per square kilometer. Also, an interaction term like this supposes that there is a multiplicative effect such that \textnormal{conflict area} $\times$ \textnormal{conflict distance} is correlated with economic growth in a way that \textnormal{conflict area} $+$ \textnormal{conflict distance} is not. We reiterate that this is a very interesting avenue for further exploration but believe that we cannot address it adequately within the scope of this paper.
}}
\end{itemize}

\item Results. Conflict duration and the number of conflicts a country is facing have an unexpected effect on growth. I would see a discussion to explain this results.

\begin{itemize}
\item \textcolor{blue}{\emph{stuff}}
\end{itemize}

\item Area covered by conflict. I don’t understand why the authors use a binary measure instead of the continuous variable.

\begin{itemize}
\item \textcolor{blue}{\emph{
	When rerunning the model using the continuous variable each of the key results remain similar. We have used the continuous version of the variable in the revised draft.
}}
\end{itemize}

\item Robustness. The strategy that consists to run a six-fold cross-validation is convincing. In the same spirit, I would like to see the same exercise when one country is left out of the sample.

\begin{itemize}
\item \textcolor{blue}{\emph{stuff}}
\end{itemize}

\item Figures have to be self contained.

\begin{itemize}
\item \textcolor{blue}{\emph{stuff}}
\end{itemize}

\item The visual presentation of the results are interesting but the classical way to present results (through tables) is a requirement.

\begin{itemize}
\item \textcolor{blue}{\emph{stuff}}
\end{itemize}

\item The recent works on cities of Quoc-Anh Do seems to me as a complement to this paper.

\begin{itemize}
\item \textcolor{blue}{\emph{stuff}}
\end{itemize}

\end{enumerate}